

\subsection{The Axis-Environments}
There is an axis environment for linear scaling, two for semi-logarithmic scaling and one for double-logarithmic scaling.
\begin{environment}{{tikzpicture}\oarg{options of tikz}}
	This is the graphics environment of \Tikz. It produces a single picture and encloses also every axis.

	Instead of using the environment version, there is also a shortcut command 

	\declareandlabel{\tikz}\marg{picture content}

	which can be used alternatively.
\end{environment}

\begin{environment}{{axis}\oarg{options}}
	The axis environment for normal plots with linear axis scaling.

	The `|every linear axis|' style key can be modified with
\begin{codeexample}[code only]
\pgfplotsset{every linear axis/.append style={...}}
\end{codeexample}
to install styles specifically for linear axes. These styles can contain both \Tikz- and \PGFPlots\ options.
\end{environment}

\begin{environment}{{semilogxaxis}\oarg{options}}
The axis environment for logarithmic scaling of~$x$ and normal scaling of~$y$.
Use
\begin{codeexample}[code only]
\pgfplotsset{every semilogx axis/.append style={...}}
\end{codeexample}
to install styles specifically for the case with |xmode=log|, |ymode=normal|.

The logarithmic scaling means to apply the natural logarithm (base $e$) to each $x$ coordinate. Furthermore, ticks will be typeset as $10^{\text{\meta{exponent}}}$, see section~\ref{sec:number:printing} for more details.
\end{environment}

\begin{environment}{{semilogyaxis}\oarg{options}}
The axis environment for normal scaling of~$x$ and logarithmic scaling of~$y$,

The style `|every semilogy axis|' will be installed for each such plot.

The same remarks as for |semilogxaxis| apply here as well.
\end{environment}

\begin{environment}{{loglogaxis}\oarg{options}}
The axis environment for logarithmic scaling of both, $x$~and~$y$ axes,
As for the other axis possibilities, there is a style `|every loglog axis|' which is installed at the environment's beginning.

The same remarks as for |semilogxaxis| apply here as well.
\end{environment}

\noindent
They are all equivalent to
\begin{codeexample}[code only]
\begin{axis}[
	xmode=log|normal,
	ymode=log|normal]
...
\end{axis}
\end{codeexample}
\noindent
with properly set variables `|xmode|' and `|ymode|' (see below).

\subsection{The \protect\texttt{\protect\textbackslash addplot} Command: Coordinate Input}
\label{sec:addplot}%
\begin{codeexample}[]
\begin{tikzpicture}
\begin{axis}[ymin=0,ymax=1,enlargelimits=false]
\addplot
	[blue!80!black,fill=blue,fill opacity=0.5] 
coordinates
{(0,0.1)    (0.1,0.15)  (0.2,0.5)   (0.3,0.62)
 (0.4,0.56) (0.5,0.58)  (0.6,0.65)  (0.7,0.6)
 (0.8,0.58) (0.9,0.55)  (1,0.52)} 
|- (axis cs:0,0) -- cycle;

\addplot
	[red,fill=red!90!black,opacity=0.5]
coordinates 
{(0,0.25)   (0.1,0.27)  (0.2,0.24)  (0.3,0.24)
 (0.4,0.26) (0.5,0.3)   (0.6,0.23)  (0.7,0.2)
 (0.8,0.15) (0.9,0.1)   (1,0.1)}
|- (axis cs:0,0) -- cycle;

\addplot[green!20!black] coordinates
	{(0,0.4) (0.2,0.75) (1,0.75)};
\end{axis}
\end{tikzpicture}
\end{codeexample}

\begin{codeexample}[]
\begin{tikzpicture}
\begin{axis}
\addplot+[id=parable,domain=-5:5] 
	gnuplot{4*x**2 - 5} 
	node[pin=180:{$4x^2-5$}]{};
\end{axis}
\end{tikzpicture}
\end{codeexample}

\pgfplotsexpensiveexample
\begin{codeexample}[]
\begin{tikzpicture}
	\begin{axis}
	\addplot3[surf,domain=0:360,samples=40] 
		{sin(x)*sin(y)};	
	\end{axis}
\end{tikzpicture}
\end{codeexample}

\pgfplotsexpensiveexample
\begin{codeexample}[]
\begin{tikzpicture}
	\begin{axis}[colormap/redyellow,colorbar]
	\addplot3[surf,
		domain=0:360,samples=40] 
		{sin(x)*sin(y)};	
	\end{axis}
\end{tikzpicture}
\end{codeexample}

\pgfplotsexpensiveexample
\begin{codeexample}[]
\begin{tikzpicture}
\begin{axis}[view={60}{30}]
    \addplot3[surf,shader=flat,
		samples=20,
        domain=-1:0,y domain=0:2*pi,
        z buffer=sort]
        ({sqrt(1-x^2) * cos(deg(y))},
		 {sqrt( 1-x^2 ) * sin(deg(y))},
		 x);
\end{axis}
\end{tikzpicture}
\end{codeexample}

Inside of an axis environment, the |\addplot| command is the main user interface. It comes in two variants: |\addplot| for twodimensional visualization and \verbpdfref{\addplot3} for threedimensional visualization.

\begin{command}{\addplot\oarg{options} \meta{input data} \meta{trailing path commands};}
\label{cmd:pgfplots:addplot}
This is the main plotting command, available within each axis environment. It can be used one or more times within an axis to add plots to the current axis. There is also an \verbpdfref{\addplot3} command which is described in section~\ref{sec:3d}.

It reads point coordinates from one of the available input sources specified by \meta{input data}, updates limits, remembers \meta{options} for use in a legend (if any) and applies any necessary coordinate transformations (or logarithms).

The \meta{options} can be omitted in which case the next entry from the |cycle list| will be inserted as \meta{options}. These keys characterize the plot's type like linear interpolation, smooth plot, constant interpolation, bar plot, mesh plots, surface plots or whatever and define colors, markers and line specifications\footnote{In version 1.2.2 and earlier, there was an explicit distiction between ``behaviour'' options like error bars, domain, number of samples etc.\ and ``style options'' like color, line width, markers etc. This distiction is obsolete now, simply collect everything into \meta{options}.}\index{Behavior Options}\index{Options!Distinction Behavior, Style Options}. Plot variants like error bars, the number of samples or a sample domain can also be configured in \meta{options}.

The \meta{input data} is one of several coordinate input tools which are described in more detail below. Finally, if |\addplot| successfully processed all coordinates from \meta{input data}, it generates \Tikz\ paths to realize the drawing operations. Any \meta{trailing path commands} are appended to the final drawing command, allowing to continue the \Tikz\ path (from the last plot coordinate).

\noindent
Some more details:
\begin{itemize}
	\item The style |/pgfplots/every axis plot| will be installed at the beginning of \meta{options}. That means you can use
\begin{codeexample}[code only]
\pgfplotsset{every axis plot/.append style={...}}
\end{codeexample}
	to add options to all your plots - maybe to set line widths to |thick|. Furthermore, if you have more than one plot inside of an axis, you can also use
\begin{codeexample}[code only]
\pgfplotsset{every axis plot no 3/.append style={...}}
\end{codeexample}
	to modify options for the plot with number~$3$ only. The first plot in an axis has number~$0$.
	\item The \meta{options} are remembered for the legend. They are available as `\declareandlabel{current plot style}' as long as the path is not yet finished or in associated error bars.
	\item See subsection~\ref{sec:markers} for a list of available markers and line styles.
	\item For log plots, \PGFPlots\ will compute the natural logarithm $\log(\cdot)$ numerically using a floating point unit developed for this purpose\footnote{This floating point unit is available as \Tikz\ library as part of \Tikz.}. For example, the following numbers are valid input to |\addplot|.
\begin{codeexample}[]
\begin{tikzpicture}
\begin{loglogaxis}
\addplot coordinates {
	(769,   1.6227e-04)
	(1793,  4.4425e-05)
	(4097,  1.2071e-05)
	(9217,  3.2610e-06)
	(2.2e5, 2.1E-6)
	(1e6,   0.00003341)
	(2.3e7, 0.00131415)
};
\end{loglogaxis}
\end{tikzpicture}
\end{codeexample}
	You can represent arbitrarily small or very large numbers as long as its logarithm can be represented as a \TeX-length (up to about~$16384$). Of course, any coordinate~$x\le 0$ is not possible since the logarithm of a non-positive number is not defined. Such coordinates will be skipped automatically (using the initial configuration |unbounded coords=discard|).

	\item For normal (non--logarithmic) axes, \PGFPlots\ applies floating point arithmetics to support large or small numbers like 0.00000001234 or $1.234\cdot 10^{24}$. Its number range is much larger than \TeX's native support for numbers. The relative precision is between $4$ and $7$ significant decimal digits for the mantisse.
	
	As soon as the axes limits are completely known, \PGFPlots\ applies a transformation which maps these floating point numbers into \TeX-precision using transformations
		\[ T_x(x) = 10^{s_x} \cdot x - a_x
		\text{ and } T_y(y) = 10^{s_y} \cdot y - a_y 
		\text{ and (for 3D plots) } T_z(y) = 10^{s_z} \cdot z - a_z 
		\]
	with properly chosen integers $s_x, s_y, s_z \in \Z$ and shifts $a_x,a_y, a_z\in \R$. Section~\ref{sec:disabledatascaling} contains a description of |disabledatascaling| and provides more details about the transformation.
	\index{Accuracy!Floating Point in \PGFPlots}%

	\item Some of the coordinate input routines use the powerful |\pgfmathparse| feature of \pgfname\ to read their coordinates, among them |plot coordinates|, |plot expression| and |plot table|. This allows to use mathematical expressions as coordinates which will be evaluated using the floating point routines (this applies to logarithmic and linear scales).
	
	\item If you did not specify axis limits manually, |\addplot| will compute them automatically. 
	The automatic computation of axis limits works as follows:
		\begin{enumerate}
			\item Every coordinate will be checked. Care has been taken to avoid \TeX's limited numerical capabilities.
			\item Since more than one |\addplot| command may be used inside of |\begin{axis}...\end{axis}|, all drawing commands will be postponed until |\end{axis}|.
		\end{enumerate}
\end{itemize}
\end{command}

\begin{addplot+}
Does the same like |\addplot[|\meta{options}|] ...;| except that \meta{options} are \emph{appended} to the arguments which would have been taken for |\addplot ...| (the element of the default list).

Thus, you can combine |cycle list| and \meta{options}.

\begin{codeexample}[]
\begin{tikzpicture}
\begin{axis}
\addplot {sin(deg(x))};
\end{axis}
\end{tikzpicture}

\begin{tikzpicture}
\begin{axis}
\addplot+[only marks] {sin(deg(x))};
\end{axis}
\end{tikzpicture}
\end{codeexample}
\end{addplot+}

\subsubsection{Coordinate Lists}
\label{pgfplots:providing:input}

\begin{addplotoperation}[]{coordinates}{\marg{coordinate list}}
\label{pgfplots:addplot:coordinates}
The `|plot coordinates|' command is like that provided by \Tikz\ and reads its input data from a sequence of point coordinates, encapsulated in round braces.
\begin{codeexample}[]
\begin{tikzpicture}
\begin{axis}
\addplot coordinates {
	(0,0)
	(0.5,1)
	(1,2)
};
\end{axis}
\end{tikzpicture}
\end{codeexample}
\noindent The coordinates can be numbers, but they can also contain mathematical expressions like |sin(0.5)| or |\h*8| (assuming you defined |\h| somewhere). However, expressions which involve round braces need to be encapsulated in a further set of curly braces, for example |({sin(0.5)},{cos(0.1)})|.

You can also supply error coordinates (reliability bounds) if you are interested in error bars. Simply append the error coordinates with `\declareandlabel{+-} \parg{ex,ey}' (or |+- |\parg{ex,ey,ez}) to the associated coordinate:
\begin{codeexample}[]
\begin{tikzpicture}
\begin{axis}
\addplot+[error bars/.cd,x dir=both,x explicit] 
coordinates {
	(0,0)   +- (0.1,0)
	(0.5,1) +- (0.4,0.2)
	(1,2)
	(2,5)   +- (1,0.1)
};
\end{axis}
\end{tikzpicture}
\end{codeexample}
or 
\begin{codeexample}[code only]
\addplot coordinates {
	 (900,1e-6) +- (0.1,0.2)
	(2600,5e-7) +- (0.2,0.5)
	(4000,7e-8) +- (0.1,0.01)
};
\end{codeexample}
These error coordinates are only used in case of error bars, see section~\ref{sec:errorbars}. You will also need to configure whether these values denote absolute or relative errors.

The coordinates as such can be numbers as |+5|, |-1.2345e3|, |35.0e2|, |0.00000123| or |1e2345e-8|. They are not limited to \TeX's precision.

Furthermore, |coordinates| allows to define ``meta data'' for each coordinate. The interpretation of meta data depends on the visualization technique: for scatter plots, meta data can be used to define colors or style associations for every point (see page~\pageref{pgfplots:scatterclasses} for an example). Meta data (if any) must be provided after the coordinates and after error bar bounds (if any) in square brackets:
\begin{codeexample}[]
\begin{tikzpicture}
\begin{axis}
\addplot+[scatter,scatter src=explicit] coordinates {
	 (900,1e-6) [1]
    (2600,5e-7) [2]
    (4000,7e-8) [3]
};
\end{axis}
\end{tikzpicture}
\end{codeexample}
Please refer to the documentation of |point meta| on page~\pageref{pgfplots:point:meta} for more information about per point meta data.
\end{addplotoperation}

\begin{pgfplotskey}{plot coordinates/math parser=\mchoice{true,false} (initially true)}
	Allows to turn off support for mathematical expressions in every coordinate inside of |plot coordinates|. This might be necessary if coordinates are not in numerical form (or if you'd like to improve speed).

	It is necessary to disable |plot coordinates/math parser| if you use some sort of symbolic transformations (i.e. text coordinates).
\end{pgfplotskey}

\subsubsection{Reading Coordinates From Files}

\begin{addplotoperation}[]{file}{\marg{name}}
\label{pgfplots:addplot:file}
\PGFPlots\ supports two ways to read plot coordinates of external files, and one of them is similar to the \Tikz-command `|plot file|'. It is to be used like
\begin{codeexample}[code only]
\addplot file {datafile.dat};
\end{codeexample}
where \marg{name} is a text file with at least two columns which will be used as $x$ and $y$ coordinates. Lines starting with `|%|' or `|#|' are ignored. Such files are often generated by \textsc{gnuplot}:
\begin{codeexample}[code only]
#Curve 0, 20 points
#x y type
0.00000 0.00000 i
0.52632 0.50235 i
1.05263 0.86873 i
1.57895 0.99997 i
...
9.47368 -0.04889 i
10.00000 -0.54402 i
\end{codeexample}
This listing has been copied from~\cite[section~16.4]{tikz}.

Plot file accepts one optional argument,

\begin{codeexample}[code only]
\addplot file[skip first] {datafile.dat};
\end{codeexample}

\noindent
which allows to skip over a non-comment header line. This allows to read the same input files as |plot table| by skipping over column names. Please note that comment lines do not count as lines here.

The input method |plot file| can also read meta data for every coordinate. As already explained for |plot coordinates| (see above), meta data can be used to change colors or other style parameters for every marker separately. Now, if |point meta| is set to |explicit| or to |explicit symbolic| and the input method is |plot file|, one further element will be read from disk -- for every line. Meta data is always the last element which is read. See page~\pageref{pgfplots:scatter:src} for information and examples about per point meta data and page~\pageref{pgfplots:scatterclasses} for an application example using |scatter/classes|.


Plot file is very similar to |plot table|: you can achieve the same effect with
\begin{codeexample}[code only]
\addplot table[x index=0,y index=1,header=false] {datafile.dat};
\end{codeexample}
\noindent Due to its simplicity, |plot file| is slightly faster while |plot table| allows higher flexibility.

Technical note: every opened file will be protocolled into your log file.
\end{addplotoperation}

\begin{pgfplotskeylist}{%
	plot file/skip first=\mchoice{true,false} (initially false),%
	plot file/ignore first=\mchoice{true,false} (initially false)}
	The two keys can be provided as arguments to |\addplot file[|\meta{options}|] |\marg{filename}|;| to skip the first non-comment entry in the file. They are equivalent.
	If you provide them in this context, the prefix |/pgfplots/plot file| can be omitted.
\end{pgfplotskeylist}

\subsubsection{Reading Coordinates From Tables}

\begin{addplotoperation}[]{table}{\oarg{column selection}\marg{file}}
\label{pgfplots:addplot:table}
The use of `|plot table|' is similar in spirit to `|plot file|', but its flexibility is higher. Given a data file like
\begin{codeexample}[code only]
dof     L2              Lmax            maxlevel
5       8.31160034e-02  1.80007647e-01  2
17      2.54685628e-02  3.75580565e-02  3
49      7.40715288e-03  1.49212716e-02  4
129     2.10192154e-03  4.23330523e-03  5
321     5.87352989e-04  1.30668515e-03  6
769     1.62269942e-04  3.88658098e-04  7
1793    4.44248889e-05  1.12651668e-04  8
4097    1.20714122e-05  3.20339285e-05  9
9217    3.26101452e-06  8.97617707e-06  10
\end{codeexample}
one may want to plot `|dof|' versus `|L2|' or `|dof|' versus `|Lmax|'. This can be done by
\begin{codeexample}[code only]
\begin{tikzpicture}
\begin{loglogaxis}[
	xlabel=Dof,
	ylabel=$L_2$ error]
\addplot table[x=dof,y=L2] {datafile.dat};
\end{loglogaxis}
\end{tikzpicture}
\end{codeexample}
or
\begin{codeexample}[code only]
\begin{tikzpicture}
\begin{loglogaxis}[
	xlabel=Dof,
	ylabel=$L_infty$ error]
\addplot table[x=dof,y=Lmax] {datafile.dat};
\end{loglogaxis}
\end{tikzpicture}
\end{codeexample}
Alternatively, you can load the table \emph{once} and use it \emph{multiple} times:
\begin{codeexample}[code only]
\pgfplotstableread{datafile.dat}\table
...
\addplot table[x=dof,y=L2] from \table;
...
\addplot table[x=dof,y=Lmax] from \table;
...
\end{codeexample}
I am not really sure how much time can be saved, but it works anyway. As a rule of thumb, decide as follows:
\begin{enumerate}
	\item If tables contain few rows and many columns, the |from |\meta{\textbackslash macro} framework will be more efficient.
	\item If tables contain more than~$200$ data points (rows), you should always use file input (and reload if necessary).
\end{enumerate}

If you do prefer to access columns by column indices instead of column names (or your tables do not have column names), you can also use
\begin{codeexample}[code only]
\addplot table[x index=2,y index=3] {datafile.dat};
\addplot table[x=dof,y index=2] {datafile.dat};
\end{codeexample}

Summary and remarks:
\begin{itemize}
	\item Use |plot table[||x||=|\marg{column name}|,||y||=|\marg{column name}|]| to access column names. Those names are case sensitive and need to exist.
	\item Use |plot table[||x index||=|\marg{column index}|,||y index||=|\marg{column index}|]| to access column indices. Indexing starts with~$0$. You may also use an index for~$x$ and a column name for~$y$.
	\item Use |plot table[||header||=false] |\marg{file name} if your input file has no column names. Otherwise, the first non-comment line is checked for column names: if all entries are numbers, they are treated as numerical data; if one of them is not a number, all are treated as column names.
	\item It is possible to read error coordinates from tables as well. Simply add options `|x error|', `|y error|' or `|x error index|'/`|y error index|' to \marg{source columns}. See section~\ref{sec:errorbars} for details about error bars.
	\item It is possible to read per point meta data (usable in |scatter src|, see page~\pageref{pgfplots:scatter:src}) as has been discussed for |plot coordinates| and |plot file| above. The meta data column can be provided using the |meta| key (or the |meta index| key).
	\item Use |plot table[|\meta{source columns}|] from |\marg{\textbackslash macro} to use a pre--read table. Tables can be read using
\begin{codeexample}[code only]
\pgfplotstableread{datafile.dat}\macroname.
\end{codeexample}
		The keyword `|from|' can be omitted.

	\item The accepted input format of those tables is as follows:
		\begin{itemize}
			\item Columns are usually separated by white spaces (at least one tab or space).

			 If you need other column separation characters, you can use the 

			\declare{col sep}\pgfmanualpdflabel{/pgfplots/table/col sep}{}|=|\mchoice{space,tab,comma,colon,semicolon,braces} 

			option which is documented in all detail in the manual for \PGFPlotstable\ which is part of \PGFPlots.
			\item Any line starting with `\#' or `\%' is ignored.
			\item The first line will be checked if it contains numerical data. If there is a column in the first line which is \emph{no} number, the complete line is considered to be a header which contains column names. Otherwise it belongs to the numerical data and you need to access column indices instead of names.

			\item There is future support for a second header line which must start with `|$flags |'. Currently, such a line is ignored. It may be used to provide number formatting hints like precision and number format if those tables shall be typeset using |\pgfplotstabletypeset| (see the manual for \PGFPlotstable).
			\item The accepted number format is the same as for `|plot coordinates|', see above.
			\item If you omit column selectors, the default is to plot the first column against the second. That means |plot table| does exactly the same job as |plot file| for this case.
			\item If you need unbalanced columns, simply use |nan| as ``empty cell'' placeholder. These coordinates will be skipped in plots.
			\index{Unbalanced Columns}%
			\index{table@\textcolor {gray}{\texttt {plot}}\texttt { table}!Unbalanced Columns}%
		\end{itemize}
	\item It is also possible to use \textbf{mathematical expressions} together with `|plot table|'. This is documented in all detail in section~\ref{pgfplots:addplot:table:expr}, but the key idea is to use one of |x expr|, |y expr|, |z expr| or |meta expr| as in `|plot table[||x expr=\thisrow{maxlevel}+3,y=L2]|'.
	\item The enhanced column generation methods provided by \PGFPlotstable\ are also accessible for plots, see the \PGFPlotstable\ manual section ``Postprocessing Data in New Columns''.

	In this context, you should consider using the key |read completely|, see below.
	\item Technical note: every opened file will be protocolled into your log file.
\end{itemize}
\end{addplotoperation}

\begin{pgfplotskey}{table/header=\mchoice{true,false} (initially true)}
	Allows to disable header identification for |plot table|. See above.
\end{pgfplotskey}
\begin{pgfplotsxykeylist}{table/\x=\marg{column name},
	table/\x\ index=\marg{column index}}
	These keys define the sources for |plot table|. If both, column names and column indices are given, column names are preferred. Column indexing starts with~$0$. The initial setting is to use |x index=0| and |y index=1|. 

	Please note that column \emph{aliases} will be considered if unknown column names are used. Please refer to the manual of \PGFPlotstable\ which comes with this package.
\end{pgfplotsxykeylist}

\begin{pgfplotsxykeylist}{table/\x\ expr=\marg{expression},table/meta expr=\marg{expression}}
	These keys allow to combine the mathematical expression parser with file input. They are listed here to complete the list of table keys, but they are described in all detail in section~\ref{pgfplots:addplot:table:expr}.

	The key idea is to provide an \meta{expression} which depends on table data (possibly on all columns in one row). Only data within the same row can be used where columns are referenced with |\thisrow|\marg{column name} or |\thisrowno|\marg{column index}. 
	
	Please refer to section~\ref{pgfplots:addplot:table:expr} for details.
\end{pgfplotsxykeylist}


\begin{pgfplotsxykeylist}{%
	table/\x\ error=\marg{column name},
	table/\x\ error index=\marg{column index},
	table/\x\ error expr=\marg{math expression}}
	These keys define input sources for error bars with explicit error values. 
	
	The |x error| method provides an input column name (or alias), the |x error index| method provides input column \emph{indices} and |x error expr| works just as |table/x expr|: it allows arbitrary mathematical expressions which may depend on any number of table columns using |\thisrow|\marg{col name}.

	Please see section~\ref{sec:errorbars} for details about the usage of error bars.
\end{pgfplotsxykeylist}
\begin{pgfplotsxykeylist}{%
	table/meta=\marg{column name},
	table/meta index=\marg{column index}}
	These keys define input sources for per point meta data. Please see page~\pageref{pgfplots:scatter:src} for details about meta data or the documentation for |plot coordinates| and |plot file| for further information.
\end{pgfplotsxykeylist}
\begin{key}{/pgfplots/table/col sep=\mchoice{space,tab,comma,semicolon,colon,braces} (initially space)}
	Allows to choose column separators for |plot table|. Please refer to the manual of \PGFPlotstable\ which comes with this package for details about |col sep|.
\end{key}
\begin{key}{/pgfplots/table/read completely=\marg{true,false} (initially false)}
	Allows to customize \verbpdfref{\addplot table}\marg{file name} such that it always reads the entire table into memory.

	This key has just one purpose, namely to create postprocessing columns on-the-fly and to plot those columns afterwards. This ``lazy evaluation'' which creates missing columns on-the-fly is documented in the \PGFPlotstable\ manual (in section ``Postprocessing Data in New Columns'').

	\paragraph{Attention:} Usually, \verbpdfref{\addplot table} only picks required entries, requiring linear runtime complexity. As soon as |read completely| is activated, tables are loaded completely into memory. Due to datastructures issues (``macro append runtime''), the runtime complexity for |read completely| is $O(N^2)$ where $N$ is the number of rows. Thus: use this feature only for ``small'' tables\footnote{This remark might be deprecated; many of the slow routines have been optimized in the meantime to have at least pseudo-linear runtime.}.
\end{key}

\subsubsection{Computing Coordinates with Mathematical Expressions}

\begin{addplotoperation}[]{\marg{math expression}}{}
\pgfmanualpdflabel{plot expression}{}
\pgfmanualpdflabel{\textbackslash addplot expression}{}
	This input method allows to provide mathematical expressions which will be sampled. But unlike |plot gnuplot|, the expressions are evaluated using the math parser of \PGF, no external program is required.

	Plot expression samples |x| from the interval $[a,b]$ where $a$ and $b$ are specified with the |domain| key. The number of samples can be configured with |samples=|\meta{N} as for plot gnuplot.

\begin{codeexample}[]
\begin{tikzpicture}
\begin{axis}
	\addplot {x^2 + 4};
	\addplot {-5*x^3 - x^2};
\end{axis}
\end{tikzpicture}
\end{codeexample}

Please note that \PGF's math parser uses degrees for trigonometric functions:
\begin{codeexample}[]
\begin{tikzpicture}
\begin{axis}
	\addplot+[domain=0:360]
		{sin(x)};
\end{axis}
\end{tikzpicture}
\end{codeexample}
\noindent If you want to use radians, use 
\begin{codeexample}[]
\begin{tikzpicture}
\begin{axis}
	\addplot+[domain=-pi:pi] 
		{sin(deg(x))};
\end{axis}
\end{tikzpicture}
\end{codeexample}
\noindent to convert the radians to degrees. The plot expression parser also accepts some more options like |samples at=|\marg{coordinate list} or |domain=|\meta{first}|:|\meta{last} which are described below.

\paragraph{Remarks}
\begin{enumerate} 
	\item What really goes on is a loop which assigns the current sample coordinate to the macro |\x|. \PGFPlots\ defines a math constant |x| which always has the same value as |\x|.

	In short: it is the same whether you write |\x| or just |x| inside of math expressions.

	The variable name can be customized using |variable=\t| (the backslash is necessary!). Then, |t| will be the same as |\t|.
\index{x@\texttt{\textbackslash x} In Coordinate Expressions}%
%\index{y@\texttt{\textbackslash y} In Coordinate Expressions}%

	\item The complete set of math expressions can be found in the \PGF\ manual. The most important mathematical operations are
	|+|, |-|, |*|, |/|, |abs|, |round|, |floor|, |mod|, |<|, |>|, |max|, |min|, |sin|, |cos|, |tan|, |deg| (conversion from radians to degrees), |rad| (conversion from degrees to radians), |atan|, |asin|, |acos|, |cot|, |sec|, |cosec|, |exp|, |ln|, |sqrt|, the constanst |pi| and |e|, |^| (power operation), |factorial|\footnote{Starting with \PGF\ versions newer than $2.00$, you can use the postfix operator \texttt{!} instead of \texttt{factorial}.}, |rand| (random between $-1$ and $1$), |rnd| (random between $0$ and $1$), number format conversions |hex|, |Hex|, |oct|, |bin| and some more. The math parser has been written by Mark Wibrow and Till Tantau~\cite{tikz}, the FPU routines have been developed as part of \PGFPlots. The documentation for both parts can be found in~\cite{tikz}.
	
	Please note, however, that trigonometric functions are defined in degrees. The character `|^|' is used for exponentiation (not `|**|' as in gnuplot).

	\item If the $x$ axis is logarithmic, samples will be drawn logarithmically.

	\item Please note that plot expression does not allow per point meta data (color data).
\end{enumerate}

\paragraph{About the precision and number range:}
\index{Accuracy!High Precision for Plot Expression}%
\index{Errors!dimension too large}%
	\index{Precision}\index{Floating Point Unit} Starting with version 1.2, |plot expression| uses a floating point unit. The FPU provides the full data range of scientific computing with a relative precision between $10^{-4}$ and $10^{-6}$. The |/pgf/fpu| key provides some more details. 

	In case the |fpu| does not provide the desired mathematical function or is too slow\footnote{Or in case you find a bug$\dotsc$}, you should consider using the |plot gnuplot| method which invokes the external, freely available program |gnuplot| as desktop calculator. 

\begin{codeexample}[]
\begin{tikzpicture}
	\begin{loglogaxis}[
		title={$\frac{1}{x^2}$}]
	\addplot[blue,domain=1:1e30] 
		{x^-2};
	\end{loglogaxis}
\end{tikzpicture}
\end{codeexample}

\begin{codeexample}[]
\begin{tikzpicture}
	\begin{semilogyaxis}[
		title={$e^x$ logarithmically plotted}]
	\addplot[blue,domain=1:700] 
		{exp(x)};
	\end{semilogyaxis}
\end{tikzpicture}
\end{codeexample}
\end{addplotoperation}

\begin{addplotoperation}[]{expression}{\marg{math expr}}
	The syntax

	|\addplot |\marg{math expression}|;|

	as short-hand equivalent for

	|\addplot expression |\marg{math expression}|;|
\end{addplotoperation}

\begin{addplotoperation}[]{(\meta{$x$ expression},\meta{$y$ expression})}{}
	A variant of \verbpdfref{\addplot expression} which allows to provide different coordinate expressions for the $x$ and $y$ coordinates. This can be used to generate parameterized plots.

	Please note that |\addplot (x,x^2)| is equivalent to |\addplot expression {x^2}|.

	Note further that since the complete point expression is surrounded by round braces, round braces for either \meta{$x$ expression} or \meta{$y$ expression} need special attention. You will need to introduce curly braces additionally to allow round braces:

	|\addplot (|\marg{$x$ expr}|, |\marg{$y$ expr}|, |\marg{$z$ expr}|);|
\end{addplotoperation}

\begin{pgfplotskeylist}{%
	domain=\meta{$x_1$}:\meta{$x_2$} (initially [-5:5]),%
	y domain=\meta{$y_1$}:\meta{$y_2$},
	domain y=\meta{$y_1$}:\meta{$y_2$}}
	Sets the function's domain(s) for |plot expression| and |plot gnuplot|. Two dimensional plot expressions are defined as functions $f\colon [x_1,x_2] \to \R$ and \meta{$x_1$} and \meta{$x_2$} are set with |domain|. Three dimensional plot expressions use functions $f\colon [x_1,x_2] \times [y_1,y_2] \to \R$ and \meta{$y_1$} and \meta{$y_2$} are set with |y domain|. If |y domain| is empty, $[y_1,y_2] = [x_1,x_2]$ is assumed for three dimensional plots (see page~\pageref{cmd:addplot3:expr} for details about three dimensional plot expressions).

	The keys |y domain| and |domain y| are the same.


	 The |domain| key won't be used if |samples at| is specified; |samples at| has higher precedence.


	 Please note that |domain| is not necessarily the same as the axis limits (which are configured with the |xmin|/|xmax| options). 
	 
	 The |domain| keys are \emph{only} relevant for |gnuplot| and |plot expression|. In case you'd like to plot only a subset of other coordinate input routines, consider using the coordinate filter |restrict x to domain|.

	 \paragraph{Remark for \Tikz-users:} |/pgfplots/domain| and |/tikz/domain| are independent options. Please prefer the \PGFPlots\ variant (i.e.\ provide |domain| to an axis, |\pgfplotsset| or a plot command). Since older versions also accepted something like |\begin{tikzpicture}[domain=|$\dotsc$|]|, this syntax is also accepted as long as no \PGFPlots\ |domain| key is set.
\end{pgfplotskeylist}

\begin{pgfplotskeylist}{%
	samples=\marg{number} (initially 25),%
	samples y=\marg{number}}
	 Sets the number of sample points for |plot expression| and |plot gnuplot|. The |samples| key defines the number of samples used for line plots while the |samples y| key is used for mesh plots (three dimensional visualisation, see page~\pageref{cmd:addplot3:expr} for details). If |samples y| is not set explicitly, it uses the value of |samples|.

	 The |samples| key won't be used if |samples at| is specified; |samples at| has higher precedence.

	The same special treatment of |/tikz/samples| and |/pgfplots/samples| as for the |domain| key applies here. See above for details.
\end{pgfplotskeylist}

\begin{pgfplotskey}{samples at=\marg{coordinate list}}
	Sets the $x$ coordinates for |plot expression| explicitly. This overrides |domain| and |samples|.

	The \marg{coordinate list} is a |\foreach| expression, that means it can contain a simple list of coordinates (comma--separated) but also complex |...| expressions like\footnote{Unfortunately, the \texttt{...} is somewhat restrictive when it comes to extended accuracy. So, if you have particularly small or large numbers (or a small distance), you have to provide a comma--separated list (or use the \texttt{domain} key).}
\begin{codeexample}[code only]
\pgfplotsset{samples at={5e-5,7e-5,10e-5,12e-5}}
\pgfplotsset{samples at={-5,-4.5,...,5}}
\pgfplotsset{samples at={-5,-3,-1,-0.5,0,...,5}}
\end{codeexample}

	The same special treatment of |/tikz/samples at| and |/pgfplots/samples at| as for the |domain| key applies here. See above for details.

	\paragraph{Attention:} |samples at| overrides |domain|, even if |domain| has been set \emph{after} |samples at|! Use |samples at={}| to clear \marg{coordinate list} and re-activate |domain|.
\end{pgfplotskey}

\subsubsection{Mathematical Expressions And File Data}
\PGFPlots\ allows to combine `|plot table|' and `|plot expression|' to get both, file input and modifications by means of mathematical expressions.

\begin{addplotoperation}[]{table}{\oarg{column selection and expressions}\marg{file}}
\label{pgfplots:addplot:table:expr}
	Besides the already discussed possibility to provide a column selection by means of column names (|x||=|\meta{name} or |x index||=|\meta{index}, see section~\ref{pgfplots:addplot:table}), it is also possible to provide mathematical expressions as arguments.

	Mathematical expressions are specified with |x expr||=|\meta{expression} inside of \meta{column selection and expressions}. They can depend on zero, one or more columns of the input file. A column is referenced using the special command `|\thisrow|\marg{column name}' within \meta{expression} (or |\thisrowno|\meta{column index}).
\begin{codeexample}[vbox]
\pgfplotstabletypeset[columns={maxlevel,L2}]{plotdata/newexperiment1.dat}

\begin{tikzpicture}
	\begin{semilogyaxis}[
		xlabel=\texttt{maxlevel}$ + 10$
	]
	\addplot table
		[x expr=\thisrow{maxlevel}+10, y=L2] 
		{plotdata/newexperiment1.dat};
	\end{semilogyaxis}
\end{tikzpicture}
\end{codeexample}

	Besides |x expr|, there are keys |y expr|, |z expr| and |meta expr| where the latter allows to provide point meta data (which is used as |scatter src| or color data for surface plots etc.).

	Inside of \meta{expression}, the following macros can be used to access numerical data cells inside of the input file:

	\begin{command}{\thisrow\marg{column name}}
		Yields the value of the column designated by \marg{column name}. There is no limit on the number of columns which can be part of a mathematical expression, but only values inside of the currently processed \emph{table row} can be used.
		
		It is possible to provide column aliases for \marg{column name} as described in the manual of \PGFPlotstable. 

		The argument \meta{column name} has to denote either an existing column or one for which a column alias exists (see the manual of \PGFPlotstable). If it can't be resolved, the math parser yields an ``Unknown function'' error message.
	\end{command}
	\begin{command}{\thisrowno\marg{column index}}
		Similar to |\thisrow|, this command yields the value of the column with index \marg{column index} (starting with $0$). 
	\end{command}
	\begin{command}{\coordindex}
		Yields the current index of the table row (starting with $0$). This does \emph{not} count header or comment lines.		
	\end{command}
	\begin{command}{\lineno}
		Yields the current line number (starting with $0$). This does also count header and comment lines.
	\end{command}

	If |x index|, |x| and |x expr| (or the corresponding keys for |y|, |z| or |meta|) are combined, this is how they interact:
	\begin{enumerate}
		\item Column access via |x| has higher precedence than index access via |x index|.
		\item Even if |x expr| is provided, the values of |x index| and |x| are still checked. Any value found using column name access or column index access is made available as |\columnx| (or |\columny|, |\columnz|, |\columnmeta|, resp.). However, the result of |x expr| is used as plot coordinate.

		This allows to access the cell values identified by |x| or |x index| using the ``pointer'' |\columnx|. I am not sure if this yields any advantage, but it is possible nevertheless. If in doubt, prefer using |\thisrow|\marg{column name}.
	\end{enumerate}

	\paragraph{Attention:} If your table has less rows than two, you may need to set |x index={},y index={}| explicitly. This is a consequence of the fact that column name/index access is still applied even if an expression is provided.
\end{addplotoperation}

\subsubsection{Computing Coordinates with Mathematical Expressions (gnuplot)}

\begin{addplotoperation}[]{gnuplot}{\oarg{further options}\marg{gnuplot code}}
In contrast to |plot expression|, the |plot gnuplot| command\footnote{Note that |plot gnuplot| is actually a re-implementation of the |plot function| method known from \PGF. It also invokes \PGF\ basic layer commands.} employs the external program |gnuplot| to compute coordinates. The resulting coordinates are written to a text file which will be plotted with |plot file|. \PGF\ checks whether coordinates need to be re-generated and calls |gnuplot| whenever necessary (this is usually the case if you change the number of samples, the argument to |plot gnuplot| or the plotted domain\footnote{Please note that \PGFPlots\ produces slightly different files than \Tikz\ when used with \texttt{plot gnuplot} (it configures high precision output). You should use different \texttt{id} for \PGFPlots\ and \Tikz\ to avoid conflicts in such a case.}).

The differences between |plot expression| and |plot gnuplot| are:
\begin{itemize}
	\item |plot expression| does not require any external programs and requires no additional command line options.
	\item |plot expression| does not produce a lot of temporary files.
	\item |plot gnuplot| uses radians for trigonometric functions while |plot expression| has degrees.
	\item |plot gnuplot| is faster.
	\item |plot gnuplot| has a larger mathematical library.
	\item |plot gnuplot| has a higher accuracy. However, starting with version 1.2, this is no longer a great problem. The new floating point unit for \TeX\ provides reasonable accuracy and the same data range as |gnuplot|.
\end{itemize}

Since system calls are a potential danger, they need to be enabled explicitly using command line options, for example
\begin{codeexample}[code only]
pdflatex -shell-escape filename.tex.
\end{codeexample}
Sometimes it is called |shell-escape| or |enable-write18|. Sometimes one needs two slashes -- that all depends on your \TeX\ distribution.
\begin{codeexample}[]
\begin{tikzpicture}
\begin{axis}
\addplot 
	gnuplot[id=sin]{sin(x)};
\end{axis}
\end{tikzpicture}
\end{codeexample}

\begin{codeexample}[]
\begin{tikzpicture}
\begin{semilogyaxis}
\addplot gnuplot
	[id=exp,domain=0:10]{exp(x)};
\end{semilogyaxis}
\end{tikzpicture}
\end{codeexample}

The \meta{options} determine the appearance of the plotted function; these parameters also affect the legend. There is also a set of options which are specific to the gnuplot interface. These options are described in all detail in \cite[section~18.6]{tikz}. A short summary is shown below.

Please note that |plot gnuplot| does not allow per point meta data (color data for each coordinate).

Please refer to \cite[section~18.6]{tikz} for more details about |plot function| and the |gnuplot| interaction.

\begin{pgfplotskey}{translate gnuplot=\mchoice{true,false} (initially true)}
	Enables or disables automatic translation of the exponentiation operator `|^|' to `|**|'. 

	This features allows to use |^| in |plot gnuplot| instead of gnuplot's |**|.
\end{pgfplotskey}
\end{addplotoperation}

\begin{addplotoperation}[]{function}{\marg{gnuplot code}}
	Use

	|\addplot function |\marg{gnuplot code}|;|

	as alias for

	|\addplot gnuplot |\marg{gnuplot code}|;|
\end{addplotoperation}

\begin{key}{/tikz/id=\marg{unique string identifier}}
	 A unique identifier for the current plot. It is used to generate temporary filenames for |gnuplot| output.
\end{key}

\begin{key}{/tikz/prefix=\marg{file name prefix}}
	 A common path prefix for temporary filenames (see \cite[section~18.6]{tikz} for details).
\end{key}

\begin{key}{/tikz/raw gnuplot}
	 Disables the use of |samples| and |domain|.
\end{key}

\subsubsection{Computing Coordinates with External Programs (shell)}

\begin{addplotoperation}[]{shell}{\oarg{further options}\marg{shell commands}}
{\small \emph{An extension by Stefan Tibus}}

In contrast to |plot gnuplot|, the |plot shell| command allows execution of arbitrary shell commands to compute coordinates. The resulting coordinates are written to a text file which will be plotted with |plot file|. \PGF\ checks whether coordinates need to be re-generated and executes the \meta{shell commands} whenever necessary.

Since system calls are a potential danger, they need to be enabled explicitly using command line options, for example
\begin{codeexample}[code only]
pdflatex -shell-escape filename.tex.
\end{codeexample}
Sometimes it is called |shell-escape| or |enable-write18|. Sometimes one needs two slashes -- that all depends on your \TeX\ distribution.
\begin{codeexample}[]
\begin{tikzpicture}
\begin{axis}
\addplot
	shell[prefix=pgfshell_,id=cos]{awk 'BEGIN{
		pi=3.14159; N=10;
		for(i=0;i<=N;i++) print i,cos(i/N*pi);}'};
\end{axis}
\end{tikzpicture}
\end{codeexample}

\begin{codeexample}[]
\begin{tikzpicture}
\begin{axis}
\addplot+[prefix=pgfshell_,id=replot]
	shell{cat pgfshell_cos.out};
	% just reprint the result from above
\end{axis}
\end{tikzpicture}
\end{codeexample}

The \meta{options} determine the appearance of the plotted function; these parameters also affect the legend. There is also a set of options which are specific to the gnuplot and the shell interface. These options are described in all detail in \cite[section~19.6]{tikz}. A short summary is shown below.
\end{addplotoperation}

\begin{key}{/tikz/id=\marg{unique string identifier}}
	 A unique identifier for the current plot. It is used to generate temporary filenames for |shell| output.
\end{key}

\begin{key}{/tikz/prefix=\marg{file name prefix}}
	 A common path prefix for temporary filenames (see \cite[section~19.6]{tikz} for details).
\end{key}

\subsubsection{Using External Graphics as Plot Sources}

\begin{addplotoperation}[]{graphics}{\marg{file name}}
\pgfkeys{/pdflinks/search key prefixes in/.add={/pgfplots/plot graphics/,}{}}
	This plot type allows to extend the plotting capabilities of \PGFPlots\ beyond its own limitations. The idea is to generate the graphics as such (for example, a contour plot, a complicated shaded surface\footnote{See also section~\ref{sec:pgfplots:surfplots} for an overview of \PGFPlots\ methods to draw shaded surfaces.} or a large point cluster) with an external program like Matlab (tm) or |gnuplot|. The graphics, however, should \emph{not} contain an axis or descriptions. Then, we use |\includegraphics| and an \PGFPlots\ axis which fits exactly on top of the imported graphics.

	Of course, one could do this manually by providing proper scales and such. The operation |plot graphics| is intended so simplify this process. However the \emph{main difficulty} is to get images with correct bounding box. Typically, you will have to adjust bounding boxes manually.

	Let's start with an example: Suppose we use, for example, matlab to generate a surface plot like
\begin{codeexample}[code only]
[X,Y] = meshgrid( linspace(-3,3,500) );
surf( X,Y, exp(-(X - Y).^2 - X.^2 ) );
shading flat; view(0,90); axis off; 
print -dpng external1
\end{codeexample}
	\noindent which is then found in |external1.png|. The |surf| command of Matlab generates the surface, the following commands disable the axis descriptions, initialise the desired view and export it. Viewing the image in any image tool, we see a lot of white space around the surface -- Matlab has a particular weakness in producing tight bounding boxes, as far as I know. Well, no problem: use your favorite image editor and crop the image (most image editors can do this automatically). We could use the free ImageMagick command
	
	|convert -trim external1.png external1.png|

	to get a tight bounding box. Then, we use

\begin{codeexample}[]
\begin{tikzpicture}
	\begin{axis}[enlargelimits=false,axis on top]
		\addplot graphics
			[xmin=-3,xmax=3,ymin=-3,ymax=3] 
			{external1};
	\end{axis}
\end{tikzpicture}
\end{codeexample}
\noindent to load the graphics\footnote{Please note that I don't have a Matlab license, so I used \texttt{gnuplot} to produce an equivalent replacement graphics.} just as if we would have drawn it with \PGFPlots. The |axis on top| simply tells \PGFPlots\ to draw the axis on top of any plots (see its description).

Please note that \PGFPlots\ offers support for smaller surface plots as well which might be an option -- unless the number of samples is too large. See section~\ref{sec:pgfplots:surfplots} for details.

\noindent However, external programs have the following advantages here: they are faster, allow more complexity and provide real $z$ buffering which is currently only simulated by \PGFPlots. Thus, it may help to consider |plot graphics| for complicated surface plots.

Our first test was successful -- and not difficult at all because graphics programs can automatically compute the bounding box. There are a couple of free tools available which can compute tight bounding boxes for |.eps| or |.pdf| graphics:
\begin{enumerate}
	\item The free vector graphics program |inkscape| can help here. Its feature ``File $\gg$ Document Properties: Fit page to selection'' computes a tight bounding box around every picture element. 

	However, some images may contain a rectangular path which is as large as the bounding box (Matlab (tm) computes such |.eps| images). In this case, use the ``Ungroup'' method (context menu of |inkscape|) as often as necessary and remove such a path.

	Finally, save as |.eps|.

	However, |inkscape| appears to have problems with postscript fonts -- it substitutes them. This doesn't pose problems in this application because fonts shouldn't be part of such images -- the descriptions will be drawn by \PGFPlots.

	\item The tool |pdfcrop| removes surrounding whitespace in |.pdf| images and produces quite good bounding boxes.
\end{enumerate}

\paragraph{Adjusting bounding boxes manually}
In case you don't have tools at hand to provide correct bounding boxes, you can still use \TeX\ to set the bounding box manually. Some viewers like |gv| provide access to low--level image coordinates. The idea is to determine the number of units which need to be removed and communicate these units to |\includegraphics|.

I am aware of the following methods to determine bounding boxes manually:
\begin{description}
	\item[inkscape] I am pretty sure that |inkscape| can do it.
	\item[gv] The ghost script viewer |gv| always shows the postscript units under the mouse cursor.
	\item[gimp] The graphics program |gimp| usually shows the cursor position in pixels, but it can be configured to display postscript points  (|pt|) instead.
\end{description}

Let's follow this approach in a further example. 

	We use |gnuplot| to draw a (relatively stupid) example data set. The gnuplot script
\begin{codeexample}[code only]
set samples 30000
set parametric
unset border
unset xtics
unset ytics
set output "external2.eps"
set terminal postscript eps color
plot [t=0:1] rand(0),rand(0) with dots notitle lw 5
\end{codeexample}
\noindent generates |external2.eps| with a uniform random sample of size $30000$. As before, we import this scatter plot into \PGFPlots\ using |plot graphics|. Again, the bounding box is too large, so we need to adjust it (|gnuplot| can do this automatically, but we do it anyway to explain the mechanisms):

Using |gv|, I determined that the bounding box needs to be shifted |12| units to the left and |9| down. Furthermore, the right end is |12| units too far off and the top area has about |8| units space wasted. This can be provided to the |trim| option of |\includegraphics|, and we use |clip| to clip the rest away: 
\begin{codeexample}[]
\begin{tikzpicture}
	\begin{axis}[axis on top,title=Graphics Import]
		\addplot graphics
			[xmin=0,xmax=1,ymin=0,ymax=1,
			% trim=left bottom right top
			includegraphics={trim=12 9 12 8,clip}]
			{external2};
		\addplot coordinates {(0,0) (1,1)};
	\end{axis}
\end{tikzpicture}
\end{codeexample}

	So, |plot graphics| takes a graphics file along with options which can be passed to |\includegraphics|. Furthermore, it provides the information how to embed the graphics into an axis. The axis can contain any other |\addplot| command as well and will be resized properly.


\paragraph{Details about \texttt{plot graphics}:}
	The loaded graphics file is drawn with
	
	|\node[/pgfplots/plot graphics/node] {\includegraphics[|\meta{options}|]|\marg{file name}|};|

	where the |node| style is a configurable style. The node is placed at the coordinate designated by |xmin|, |ymin|. 
	
	The \meta{options} are any arguments provided to the |includegraphics| key (see below) and |width| and |height| determined such that the graphics fits exactly into the rectangle denoted by the |xmin|, |ymin| and |xmax|, |ymax| coordinates.

	The scaling will thus ignore the aspect ratio of the external image and prefer the one used by \PGFPlots. You will need to provide |width| and |height| to the \PGFPlots\ axis to change its scaling. Use the |scale only axis| key in such a case.
\end{addplotoperation}
\begin{pgfplotsxykeylist}{
	plot graphics/\x min=\marg{coordinate},
	plot graphics/\x max=\marg{coordinate}}
	These keys are required for |plot graphics| and provide information about the external data range. The graphics will be squeezed between these coordinates. The arguments are axis coordinates.
\end{pgfplotsxykeylist}

\begin{pgfplotskey}{plot graphics/includegraphics=\marg{options}}
	A list of options which will be passed as--is to |\includegraphics|. Interesting options include the \declareandlabel{trim}|=|\meta{left} \meta{bottom} \meta{right} \meta{top} key which reduces the bounding box and \pgfmanualpdflabel{/pgfplots/plot graphics/clip}{\declaretext{clip}} which discards everything outside of the bounding box. The scaling options won't have any effect, they will be overwritten by \PGFPlots.
\end{pgfplotskey}
\begin{stylekey}{/pgfplots/plot graphics/node}
	A predefined style used for the \Tikz\ node containing the graphics. The predefined value is
\begin{codeexample}[code only]
\pgfplotsset{
	plot graphics/node/.style={
		transform shape,
		inner sep=0pt,
		outer sep=0pt,
		every node/.style={},
		anchor=south west,
		at={(0pt,0pt)},
		rectangle
	}
}
\end{codeexample}
\end{stylekey}

\begin{pgfplotskey}{plot graphics}
	This key belongs to the public low--level plotting interface. You won't need it in most cases.

	This key is similar to |sharp plot| or |smooth| or |const plot|: it installs a low--level plot--handler which expects exactly two points: the lower left corner and the upper right one. The graphics will be drawn between them. The graphics file name is expected as value of the |/pgfplots/plot graphics/src| key. The other keys described above need to be set correctly (excluding the limits, these are ignored at this level of abstraction). This key can be used independently of an axis.
\end{pgfplotskey}

\begin{pgfplotskey}{plot graphics/lowlevel draw=\marg{width}\marg{height}}
	A low--level interface for |plot graphics| which actually invokes |\includegraphics|. But there is no magic involved: the command is simply expected to draw a box of dimensions \meta{width} $\times$ \meta{height}. The coordinate system has already been shifted correctly.

	The initial configuration is

	|\includegraphics[|\meta{value of ``{\normalfont\texttt{plot graphics/includegraphics}}''}|,width=#1,height=#2]|

	\hspace{10pt}\marg{value of ``{\normalfont\texttt{plot graphics/src}}''}.

	Thus, you can tweak |plot graphics| to place any \TeX\ box of the desired dimensions into an axis between the provided minimum and maximum coordinates. It is not necessary to make use of the graphics file name or the options in the `|includegraphics|' key if you overwrite this lowlevel interface with

	|plot graphics/lowlevel draw/.code 2 args=|\marg{code which depends on \texttt{\#1} and \texttt{\#2}}.
\end{pgfplotskey}
	
