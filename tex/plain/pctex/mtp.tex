% MTP.TEX  VERSION 4.0 (September 2004)
% COPYRIGHT (C) 1992, 1993, 2000, 2002, 2004 BY PUBLISH OR PERISH, INC.



%%%%%%%%%%%%%%%%%%%%%%%%%%%%%%%%%%%%%%%%%%%%%%%%%%%%%%%%%%%%%%%%%%%%%%%%%%%%%%%%%%
%% Comments in boxes, like this, are for those designing their own style files. %%
%% The first two lines of mtp.tex save the initial category codes of " and @.   %%
%% They may be deleted in your own style file, since you presumably know them.  %%
%%%%%%%%%%%%%%%%%%%%%%%%%%%%%%%%%%%%%%%%%%%%%%%%%%%%%%%%%%%%%%%%%%%%%%%%%%%%%%%%%%

\expandafter\edef\csname qqcode\string @\endcsname{\the\catcode`\"}
\expandafter\edef\csname atcode\string @\endcsname{\the\catcode`\@}

% We used \csname...\endcsname and \string @, because we can't make @ a letter
% until we've stored its old category code.

% Prevent file from being read twice, by testing if \s@b@ hasn't yet been defined
% (and subsequently let \s@b@ be type 8 _ ). 

\expandafter\ifx\csname s\string @b\string @\endcsname\relax\else\endinput\fi
\catcode`\@=11
\catcode`\"=12   
\catcode`\_=8
\let\s@b@=_

% We need _ active in math, so it can insert corrections in some subscripts.
% \space@ is a token that has been let equal to a space.

\def\space@.{\futurelet\space@\relax}
\space@. %  % We must have the space after the period, not a new line.
\catcode`\_=12
\mathcode`\_="8000
\begingroup
 \catcode`\_=\active
 \gdef_#1{\s@b@{\futurelet\next\s@b@@#1}}
\endgroup
\def\s@b@@{%
 \ifx\next\space@\def\next@. {\futurelet\next\s@b@@}\else
  \def\next@.{%
   \ifx\next f\mkern-\thr@@ mu\else
   \ifx\next j\mkern-\tw@ mu\else
   \ifx\next p\mkern-\tw@ mu\else
   \ifx\next t\mkern\@ne mu\else
   \ifx\next y\mkern-\@ne mu\else
   \ifx\next A\mkern-\tw@ mu\else
   \ifx\next B\mkern-\@ne mu\else
   \ifx\next D\mkern-\@ne mu\else
   \ifx\next H\mkern-\@ne mu\else
   \ifx\next I\mkern-\@ne mu\else
   \ifx\next K\mkern-\@ne mu\else
   \ifx\next L\mkern-\@ne mu\else
   \ifx\next M\mkern-\@ne mu\else
   \ifx\next P\mkern-\@ne mu\else
   \ifx\next X\mkern-\tw@ mu\else
   \fi\fi\fi\fi\fi\fi\fi\fi\fi\fi\fi\fi\fi\fi\fi}%
 \fi
 \next@.}

% New definitions for characters on MTMI... fonts

\mathchardef\Gamma="0180
\mathchardef\Delta="0181
\mathchardef\Theta="0182
\mathchardef\Lambda="0183
\mathchardef\Xi="0184
\mathchardef\Pi="0185
\mathchardef\Sigma="0186
\mathchardef\Upsilon="0187
\mathchardef\Phi="0188
\mathchardef\Psi="0189
\mathchardef\Omega="017F
\mathchardef\varGamma="0100
\mathchardef\varDelta="0101
\mathchardef\varTheta="0102
\mathchardef\varLambda="0103
\mathchardef\varXi="0104
\mathchardef\varPi="0105
\mathchardef\varSigma="0106
\mathchardef\varUpsilon="0107
\mathchardef\varPhi="0108
\mathchardef\varPsi="0109
\mathchardef\varOmega="010A
\mathchardef\varkappa="017E
\mathchardef\upalpha="0192
\mathchardef\upbeta="0193
\mathchardef\upgamma="0194
\mathchardef\updelta="0195
\mathchardef\upepsilon="0196
\mathchardef\upzeta="0197
\mathchardef\upeta="0198
\mathchardef\uptheta="0199
\mathchardef\upiota="019A
\mathchardef\upkappa="019B
\mathchardef\uplambda="019C
\mathchardef\upmu="019D
\mathchardef\upnu="019E
\mathchardef\upxi="019F
\mathchardef\uppi="01A0
\mathchardef\uprho="01A1
\mathchardef\upsigma="01A2
\mathchardef\uptau="01A3
\mathchardef\upupsilon="01A4
\mathchardef\upphi="01A5
\mathchardef\upchi="01A6
\mathchardef\uppsi="01A7
\mathchardef\upomega="01A8
\mathchardef\upvarepsilon="01A9
\mathchardef\upvartheta="01AA
\mathchardef\upvarpi="01AB
\mathchardef\upvarrho="01AC
\mathchardef\upvarsigma="01AD
\mathchardef\upvarphi="01AE
\mathchardef\upvarkappa="01AF
\mathchardef\varbeta="01B0
\mathchardef\upvarbeta="01B1
\mathchardef\vardelta="01B2
\mathchardef\upvardelta="01B3

\mathchardef\openclubsuit="0280
\mathchardef\shadedclubsuit="0281
\mathchardef\openspadesuit="0282
\mathchardef\shadedspadesuit="0283
\mathcode`\(="412E
\mathcode`\)="512F
\delcode`\(="12E300
\delcode`\)="12F301
\mathcode`\.="013A
\mathchardef\ldotp="613A
\mathcode`\,="613B
\mathcode`\!="518A
\mathcode`\?="518B
\mathcode`\[="418C
\mathcode`\]="518D
\mathcode`\0="7130
\mathcode`\1="7131
\mathcode`\2="7132
\mathcode`\3="7133
\mathcode`\4="7134
\mathcode`\5="7135
\mathcode`\6="7136
\mathcode`\7="7137
\mathcode`\8="7138
\mathcode`\9="7139
\def\dag{\mathhexbox18E}
\def\ddag{\mathhexbox18F}
\def\S{\mathhexbox190}
\def\P{\mathhexbox191}
\mathchardef\dagger="218E
\mathchardef\ddagger="218F

% Macro to construct \mathstrut properly (and \Mathstrut@ and \ex@ for AmS-TeX, as
% well as \newmu@, used later on to get \Hat, etc, defined properly).

\def\usingMTPsizes#1#2#3{%
 \dimen@#1\relax
 \ifx\amstexloaded@\relax
  \ht\Mathstrutbox@.75\dimen@
  \dp\Mathstrutbox@.25\dimen@
  \setboxz@h{x\dimen@.2326ex\xdef\Ex@{\the\dimen@}}%
  \ex@\Ex@
  \setboxz@h{$\mkern1mu$}\newmu@=\@M\wd\z@
 \fi
 \dimen@ii=.38\dimen@
 \dimen@=.72\dimen@
 \edef\tmathstrut@{\vrule height \the\dimen@ depth \the\dimen@ii width\z@}%
 \dimen@#2\relax
 \dimen@ii=.22\dimen@
 \dimen@=.78\dimen@
 \edef\smathstrut@{\vrule height \the\dimen@ depth \the\dimen@ii width\z@}%
 \dimen@#3\relax
 \dimen@ii=.23\dimen@
 \dimen@=.76\dimen@
 \edef\fmathstrut@{\vrule height \the\dimen@ depth \the\dimen@ii width\z@}%
 \def\mathstrut{\mathchoice{\tmathstrut@}{\tmathstrut@}{\smathstrut@}%
   {\fmathstrut@}}%
}

% Old style numbers are no longer on the math italic font. (Calligraphic letters
% are also not on the math symbol font, but we will allow calligraphic letters later.)

\let\mit=\undefined
\let\oldstyle=\undefined
\ifx\amstexloaded@\relax
 \let\oldnos=\undefined
\fi

% Definitions for characters on MTSY... fonts

\mathchardef\triangleleft="2247
\mathchardef\triangleright="2246
\def\vec{\mathaccent"0245 }
\ifx\amstexloaded@\relax
 \def\t#1#2{{\edef\next@{\the\font}\textfontii\accent"41 \next@#1#2}}
\else
 \def\t#1{{\edef\next{\the\font}\the\textfont2\accent"41\next#1}}
\fi
\mathchardef\comp="2242
\mathchardef\setdif="2258
\mathchardef\cupprod="2259
\mathchardef\capprod="225A
\mathcode`\+="2243
\mathcode`\=="3244
\mathchardef\Relbar="3248
\mathcode`\;="6249
\mathcode`\:="3257
\mathchardef\colon="6257

% Individually designed characters for ones that used to be built up from pieces

\mathchardef\hbar="0284
\mathchardef\notin="3285
\mathchardef\angle="0286
\mathchardef\doteq="3287
\mathchardef\models="3288
\mathchardef\bowtie="3289
\mathchardef\cong="328A
\mathchardef\hookleftarrow="328B
\mathchardef\hookrightarrow="328C
\mathchardef\longleftarrow="328D
\mathchardef\longrightarrow="328E
\mathchardef\Longleftarrow="328F
\mathchardef\Longrightarrow="3290
\mathchardef\mapsto="3291
\mathchardef\longmapsto="3292
\mathchardef\longleftrightarrow="3293
\mathchardef\Longleftrightarrow="3294
\mathchardef\rightleftharpoons="3295
\mathchardef\notless="3296
\let\nless=\notless
\mathchardef\notleq="3297
\let\nleq=\notleq
\mathchardef\notprec="3298
\let\nprec=\notprec
\mathchardef\notpreceq="3299
\let\npreceq=\notpreceq
\mathchardef\notsubset="329A
\let\nsubset=\notsubset
\mathchardef\notsubseteq="329B
\let\nsubseteq=\notsubseteq
\mathchardef\notsqsubseteq="329C
\let\nsqsubseteq=\notsqsubseteq
\mathchardef\notgr="329D
\let\ngtr=\notgr
\mathchardef\notgeq="329E
\let\ngeq=\notgeq
\mathchardef\notsucc="329F
\let\nsucc=\notsucc
\mathchardef\notsucceq="32A0
\let\nsucceq=\notsucceq
\mathchardef\notsupset="32A1
\let\nsupset=\notsupset
\mathchardef\notsupseteq="32A2
\let\nsupseteq=\notsupseteq
\mathchardef\notsqsupseteq="32A3
\let\nsqsupseteq=\notsqsupseteq
\mathchardef\neq="32A4 \let\ne=\neq
\mathchardef\notequiv="32A5
\mathchardef\notsim="32A6
\mathchardef\notsimeq="32A7
\mathchardef\notapprox="32A8
\mathchardef\notcong="32A9
\mathchardef\notasymp="32AA

% We redefine how ' and '' and ''', etc.,  work, so that there will be
% extra space before the first ' mark.  Also, for constructions like
% f''^3, we want extra space after the last prime.  Spacing is determined
% by counters \firstprimesep and \nonprimesep, initially set to have the
% values 2 and 7.

\newif\iffirstprime@
\newdimen\emcent@
\emcent@=.01em
\newcount\firstprimesep
\firstprimesep=2
{\catcode`\'=\active \gdef'{^\bgroup\firstprime@true\prim@s}}
\def\prim@s{\iffirstprime@\kern\firstprimesep\emcent@\fi
  \prime\firstprime@false\futurelet\next\pr@m@s}
\newcount\nonprimesep
\nonprimesep=7
\def\pr@m@s{\ifx'\next\def\next@##1{\prim@s}\else
 \ifx^\next\def\next@##1##2{\kern\nonprimesep\emcent@##2\egroup}\else
 \let\next@\egroup\fi\fi\next@}   

% The accents on MTSY... fonts are not in same position as on the text fonts.
% We store old values, in case they come from amstex, so aren't simply \mathaccent... 

\let\@grave@\grave
\let\@acute@\acute
\let\@check@\check
\let\@breve@\breve
\let\@bar@\bar
\let\@hat@\hat
\let\@dot@\dot
\let\@tilde@\tilde
\let\@ddot@\ddot
\def\grave{\ifnum\fam=\m@ne\mathaccent"024A \else\@grave@\fi}
\def\acute{\ifnum\fam=\m@ne\mathaccent"024B \else\@acute@\fi}
\def\check{\ifnum\fam=\m@ne\mathaccent"024C \else\@check@\fi}
\def\breve{\ifnum\fam=\m@ne\mathaccent"024D \else\@breve@\fi}
\def\bar{\ifnum\fam=\m@ne\mathaccent"024E \else\@bar@\fi}
\def\hat{\ifnum\fam=\m@ne\mathaccent"024F \else\@hat@\fi}
\def\dot{\ifnum\fam=\m@ne\mathaccent"0250 \else\@dot@\fi}
\def\tilde{\ifnum\fam=\m@ne\mathaccent"0251 \else\@tilde@\fi}
\def\ddot{\ifnum\fam=\m@ne\mathaccent"0252 \else\@ddot@\fi}
\def\dddot{\mathaccent"02AB }
\def\ddddot{\mathaccent"02AC }
\def\dddotup{\mathaccent"02AD}
\def\ddddotup{\mathaccent"02AE }
\mathchardef\digamma="02B1 
\def\yen{\ifmmode\mathchar"02B0\else\mathhexbox2B0\fi}
\mathchardef\hslash="02AF

% New definitions for the Ams-TeX constructions \Hat, ...  (The command
% \usingMTPsizes resets \newmu@ for fonts not loaded at their design size).

\ifx\amstexloaded@\relax
 \alloc@1\dimen\dimendef\insc@unt\newmu@ % can't say \newcount@, since that's \outer
 \newmu@5555\p@
 \def\dimentomu@{%
  \multiply\accentdimen@100
  \expandafter\getpoints@\the\accentdimen@\getpoints@
  \multiply\pointcount@18
  \divide\pointcount@\@m
  \dimen@\newmu@
  \dimen@ii5555\p@
  \divide\dimen@\dimen@ii
  \divide\pointcount@\dimen@
  \global\accentmu@\pointcount@
  }%
 \Makeacc@\Hat{24F}%
 \Makeacc@\Check{24C}%
 \Makeacc@\Tilde{251}%
 \Makeacc@\Acute{24B}%
 \Makeacc@\Grave{24A}%
 \Makeacc@\Dot{250}%
 \Makeacc@\Ddot{252}%
 \Makeacc@\Dotup{254}%
 \Makeacc@\Ddotup{255}%
 \Makeacc@\Breve{24D}%
 \Makeacc@\Bar{24E}%
 \Makeacc@\Wbar{278 }%
 \Makeacc@\Wwbar{253 }%
 \Makeacc@\What{279 }%
 \Makeacc@\Wtilde{27A }%
 \Makeacc@\Wcheck{27B }%
 \Makeacc@\Wwhat{380 }%
 \Makeacc@\Wwtilde{381 }%
 \Makeacc@\Wwcheck{37D }%
 \Makeacc@\Oacc{256}%
 \Makeacc@\Dddot{2AB }
 \Makeacc@\Ddddot{2AC }
 \Makeacc@\Dddotup{2AD}
 \Makeacc@\Ddddotup{2AE }
 \def\Vec{\relax\ifmmode\DN@{\mathaccent@{"0245 }}\else %%% Can't use \RIfM@ here!!!
  \DN@{\nonmatherr@\Vec}\fi\next@}%
\fi

% We set up the mechanism to allow use of \cal.

% We use \alloc@@ instead of \alloc@ so that data for \Calfam and \mbffam are not written
% to the log file (also can't use \newfam in definition of \usecal, since it's \outer).

\def\alloc@@#1#2#3#4#5{\global\advance\count1#1by\@ne
 \ch@ck#1#4#2\allocationnumber=\count1#1
 \global#3#5=\allocationnumber}
\newfam\mbffam
\def\usecal{%
 \ifx\Calfam\undefined
  \alloc@@8\fam\chardef\sixt@@n\Calfam
 \fi
 \def\cal{\fam\Calfam}%
 \ifx\amstexloaded@\relax
  \def\Cal@@##1{\noaccents@\fam\Calfam##1}%                     
 \fi
}

% Make sure that periods in math constructions come from the new math fonts,
% not the text fonts.

\def\vdots{\vbox{\baselineskip4\p@\lineskiplimit\z@
 \kern6\p@\hbox{$\m@th.$}\hbox{$\m@th.$}\hbox{$\m@th.$}}}
\def\ddots{\mathinner{\mkern1mu\raise7\p@\vbox{\kern7\p@
 \hbox{$\m@th.$}}\mkern2mu
 \raise4\p@\hbox{$\m@th.$}\mkern2mu\raise\p@\hbox{$\m@th.$}\mkern1mu}}

% Similarly for AmS-TeX, making sure letters come from \textfont0. 

\ifx\amstexloaded@\relax
 \def\varinjlim{\mathop{\vtop{\ialign{##\crcr
  \hfil\the\textfont\z@ lim\hfil\crcr
  \noalign{\nointerlineskip}\rightarrowfill\crcr
  \noalign{\nointerlineskip\kern-\ex@}\crcr}}}}
 \def\varprojlim{\mathop{\vtop{\ialign{##$$\crcr
  \hfil\the\textfont\z@ lim\hfil\crcr
  \noalign{\nointerlineskip}\leftarrowfill\crcr
  \noalign{\nointerlineskip\kern-\ex@}\crcr}}}}
 \def\varliminf{\mathop{\underline{\vrule height\z@ depth.2ex width\z@
  \hbox{\the\textfont\z@ lim}}}}
 \def\varlimsup{\mathop{\overline{\hbox{\the\textfont\z@ lim}}}}
 \def\spdot{^{\hbox{\raise\ex@\hbox{$\m@th.$}}}}
 \def\spddot{^{\hbox{\raise\ex@\hbox{$\m@th..$}}}}
 \def\spdddot{^{\hbox{\raise\ex@\hbox{$\m@th...$}}}}
 \def\spddddot{^{\hbox{\raise\ex@\hbox{$\m@th....$}}}}
\fi

% Allow \uproot and \leftroot with \sqrt and \root.  The position 
% of the root is more complicated than in plain TeX, mainly 
% because there are several different fonts contributing radicals.

\newcount\uproot@
\newcount\leftroot@
\ifx\amstexloaded@\relax\else  
 \def\nonmatherr@#1{\errmessage{\noexpand#1allowed only in math mode}}
\fi
\def\uproot#1{\relax\ifmmode\uproot@#1\relax\else\nonmatherr@\uproot\fi}
\def\leftroot#1{\relax\ifmmode\leftroot@#1\relax\else\nonmatherr@\leftroot\fi}
\def\UPROOT#1{\relax\ifmmode\uproot@#1\relax\else\nonmatherr@\UPROOT\fi}
\def\LEFTROOT#1{\relax\ifmmode\leftroot@#1\relax\else\nonmatherr@\LEFTROOT\fi}
\def\root#1\of#2{\setbox\rootbox\hbox{$\m@th\scriptscriptstyle{#1}$}%
 \mathpalette\r@@t{#2}}
\def\r@@t#1#2{\setbox\z@\hbox{$\uproot@\z@\leftroot\z@\m@th#1\sqrt{#2}$}%
 \dimen@\ht\z@\advance\dimen@-\dp\z@
 \dimen@ii\dimen@
  \ifdim\dimen@>30\p@\advance\dimen@ii-\sixt@@n\p@\else
  \ifdim\dimen@>24\p@\advance\dimen@ii-8\p@\else
  \ifdim\dimen@>18\p@\advance\dimen@ii-6\p@\else
  \ifdim\dimen@>12\p@\advance\dimen@ii-4\p@\else
  \ifdim\dimen@>10\p@\advance\dimen@ii-\tw@\p@\fi\fi\fi\fi\fi
 \setbox\tw@\hbox{$\m@th#1\mskip\uproot@ mu$}\advance\dimen@ii by1.667\wd\tw@
 \mkern-\leftroot@ mu\mkern5mu\raise.6\dimen@ii\copy\rootbox
 \mkern-8mu\mkern\leftroot@ mu\box\z@\leftroot\z@\uproot\z@}

% Extra symbols provided; some are on symbol fonts, some on first extension font.

\def\dotup{\mathaccent"0254 }
\def\ddotup{\mathaccent"0255 }
\def\oacc{\mathaccent"0256 }
\let\mathring=\oacc
\def\wbar{\mathaccent"0278 }
\def\wwbar{\mathaccent"0253 }
\def\what{\mathaccent"0279 }
\def\wtilde{\mathaccent"027A }
\def\wcheck{\mathaccent"027B }
\def\wwhat{\mathaccent"0380 }%
\def\wwtilde{\mathaccent"0381 }%
\def\wwcheck{\mathaccent"037D }%
\mathchardef\bigcupprod="138E
\mathchardef\bigcapprod="1390
\mathchardef\iintop="1392    \def\iint{\iintop\nolimits}
\mathchardef\iiintop="1394   \def\iiint{\iiintop\nolimits}
\mathchardef\oiintop="1396   \def\oiint{\oiintop\nolimits}
\mathchardef\oiiintop="1398  \def\oiiint{\oiiintop\nolimits}
\mathchardef\cwointop="139A  \def\cwoint{\cwointop\nolimits}
\mathchardef\awointop="139C  \def\awoint{\awointop\nolimits}
\mathchardef\cwintop="139E   \def\cwint{\cwintop\nolimits}


% Parts for braces are in different places on the first extension font
% and we use specifically designed middles, instead of trying to construct
% them out of end pieces.

\mathchardef\mbraceu="386
\mathchardef\mbraced="387
\mathchardef\braceld="382
\mathchardef\bracerd="383
\mathchardef\bracelu="384
\mathchardef\braceru="385
\def\downbracefill{$\m@th\braceld\leaders\vrule\hfill\mbraced
  \leaders\vrule\hfill\bracerd$}%
\def\upbracefill{$\m@th\bracelu\leaders\vrule\hfill\mbraceu
  \leaders\vrule\hfill\braceru$}%
\def\lmoustache{\delimiter"4382389 }
\def\rmoustache{\delimiter"538338A }
\def\lgroup{\delimiter"412E33A } 
\def\rgroup{\delimiter"512F33B } 
\def\bracevert{\delimiter"38D38D } 

% Now come the macros for the bold fonts.

\def\hexnumber@#1{\ifcase#1 0\or 1\or 2\or 3\or 4\or 5\or 6\or 7\or 8\or
 9\or A\or B\or C\or D\or E\or F\fi}

\def\bm{\futurelet\next\bm@}%
\def\bm@{\ifx\next'\def\next@##1{\bprime@}\else
 \ifcat\noexpand\next A\def\next@##1{{\fam\mtbmi@\relax\next}}\else
 \ifcat\noexpand\next0\def\next@{\bm@@}\else
 \def\next@{\errmessage{Invalid use of \string\bm}}\fi\fi\fi
 \next@}

\newcount\firstbprimesep
\firstbprimesep=2
\def\bprime@{^\bgroup\firstprime@true\bprim@s}
\def\bprim@s{\iffirstprime@\kern\firstbprimesep\emcent@\fi
  \bmprime\firstprime@false\futurelet\next\bpr@m@s}
\newcount\nonbprimesep
\nonbprimesep=7
\def\bpr@m@s{\ifx'\next\def\next@##1{\bprim@s}\else
 \ifx^\next\def\next@##1##2{\kern\nonbprimesep\emcent@##2\egroup}\else
 \let\next@\egroup\fi\fi\next@}

\newif\ifnumeral@
\newcount\codecount@

\def\bm@@#1{%
 \codecount@=`#1\relax
 \numeral@false
 \ifnum\codecount@>47 \ifnum\codecount@<58 \numeral@true\fi\fi
 \ifnumeral@
  {\fam\mtbmi@\relax#1}%
 \else
  \ifx#1+\mathchar"2\mtbsy@@43
  \else\ifx#1-\mathchar"2\mtbsy@@00
  \else\ifx#1=\mathchar"3\mtbsy@@44
  \else\ifx#1<\mathchar"3\mtbmi@@3C
  \else\ifx#1>\mathchar"3\mtbmi@@3E
  \else\ifx#1/\mathchar"0\mtbmi@@3D
  \else\ifx#1(\mathchar"4\mtbmi@@2E
  \else\ifx#1)\mathchar"5\mtbmi@@2F
  \else\ifx#1[\mathchar"4\mtbmi@@8C
  \else\ifx#1]\mathchar"5\mtbmi@@8D
  \else\ifx#1|\mathchar"0\mtbsy@@6A
  \else\ifx#1*\mathchar"2\mtbsy@@03
  \else\ifx#1.\mathchar"0\mtbmi@@3A
  \else\ifx#1,\mathchar"6\mtbmi@@3B
  \else\ifx#1;\mathchar"6\mtbsy@@49 
  \else\ifx#1:\mathchar"3\mtbsy@@57
  \else\ifx#1!\mathchar"5\mtbmi@@8A
  \else\ifx#1?\mathchar"5\mtbmi@@8B
  \else\errmessage{Invalid use of \string\bm}%
  \fi\fi\fi\fi\fi\fi\fi\fi\fi\fi\fi\fi\fi\fi\fi\fi\fi\fi
\fi}
\def\bmdefs@{%
 \mathchardef\bmvarGamma="0\mtbmi@@00
 \mathchardef\bmvarDelta="0\mtbmi@@01
 \mathchardef\bmvarTheta="0\mtbmi@@02
 \mathchardef\bmvarLambda="0\mtbmi@@03
 \mathchardef\bmvarXi="0\mtbmi@@04
 \mathchardef\bmvarPi="0\mtbmi@@05
 \mathchardef\bmvarSigma="0\mtbmi@@06
 \mathchardef\bmvarUpsilon="0\mtbmi@@07
 \mathchardef\bmvarPhi="0\mtbmi@@08
 \mathchardef\bmvarPsi="0\mtbmi@@09
 \mathchardef\bmvarOmega="0\mtbmi@@0A
 \mathchardef\bmGamma="0\mtbmi@@80
 \mathchardef\bmDelta="0\mtbmi@@81
 \mathchardef\bmTheta="0\mtbmi@@82
 \mathchardef\bmLambda="0\mtbmi@@83
 \mathchardef\bmXi="0\mtbmi@@84
 \mathchardef\bmPi="0\mtbmi@@85
 \mathchardef\bmSigma="0\mtbmi@@86
 \mathchardef\bmUpsilon="0\mtbmi@@87
 \mathchardef\bmPhi="0\mtbmi@@88
 \mathchardef\bmPsi="0\mtbmi@@89
 \mathchardef\bmOmega="0\mtbmi@@7F
 \mathchardef\bmalpha="0\mtbmi@@0B
 \mathchardef\bmbeta="0\mtbmi@@0C
 \mathchardef\bmgamma="0\mtbmi@@0D
 \mathchardef\bmdelta="0\mtbmi@@0E
 \mathchardef\bmepsilon="0\mtbmi@@0F
 \mathchardef\bmzeta="0\mtbmi@@10
 \mathchardef\bmeta="0\mtbmi@@11
 \mathchardef\bmtheta="0\mtbmi@@12
 \mathchardef\bmiota="0\mtbmi@@13
 \mathchardef\bmkappa="0\mtbmi@@14
 \mathchardef\bmlambda="0\mtbmi@@15
 \mathchardef\bmmu="0\mtbmi@@16
 \mathchardef\bmnu="0\mtbmi@@17
 \mathchardef\bmxi="0\mtbmi@@18
 \mathchardef\bmpi="0\mtbmi@@19
 \mathchardef\bmrho="0\mtbmi@@1A
 \mathchardef\bmsigma="0\mtbmi@@1B
 \mathchardef\bmtau="0\mtbmi@@1C
 \mathchardef\bmupsilon="0\mtbmi@@1D
 \mathchardef\bmphi="0\mtbmi@@1E
 \mathchardef\bmchi="0\mtbmi@@1F
 \mathchardef\bmpsi="0\mtbmi@@20
 \mathchardef\bmomega="0\mtbmi@@21
 \mathchardef\bmvarepsilon="0\mtbmi@@22
 \mathchardef\bmvartheta="0\mtbmi@@23
 \mathchardef\bmvarpi="0\mtbmi@@24
 \mathchardef\bmvarrho="0\mtbmi@@25
 \mathchardef\bmvarsigma="0\mtbmi@@26
 \mathchardef\bmvarphi="0\mtbmi@@27
 \mathchardef\bmvarkappa="0\mtbmi@@7E
 \mathchardef\bmleftharpoonup="3\mtbmi@@28
 \mathchardef\bmleftharpoondown="3\mtbmi@@29
 \mathchardef\bmrightharpoonup="3\mtbmi@@2A
 \mathchardef\bmrightharpoondown="3\mtbmi@@2B
 \def\bmlparens{\delimiter"\mtbmi@@2E\mtbex@@00 }%
 \def\bmrparens{\delimiter"\mtbmi@@2F\mtbex@@01 }%
 \def\bmslash{\delimiter"\mtbmi@@3D\mtbex@@0E }%
 \mathchardef\bmstar="2\mtbmi@@3F
 \mathchardef\bmpartial="0\mtbmi@@40
 \mathchardef\bmflat="0\mtbmi@@5B
 \mathchardef\bmnatural="0\mtbmi@@5C
 \mathchardef\bmsharp="0\mtbmi@@5D
 \mathchardef\bmsmile="3\mtbmi@@5E
 \mathchardef\bmfrown="3\mtbmi@@5F
 \mathchardef\bmell="0\mtbmi@@60
 \mathchardef\bmimath="0\mtbmi@@7B
 \mathchardef\bmjmath="0\mtbmi@@7C
 \mathchardef\bmwp="0\mtbmi@@7D
 \def\bmlbrack{\delimiter"4\mtbmi@@8C\mtbex@@02 }%
 \def\bmrbrack{\delimiter"5\mtbmi@@8D\mtbex@@03 }%
 \mathchardef\bmdagger="2\mtbmi@@8E
 \mathchardef\bmddagger="2\mtbmi@@8F
 \mathchardef\bmupalpha="0\mtbmi@@92
 \mathchardef\bmupbeta="0\mtbmi@@93
 \mathchardef\bmupgamma="0\mtbmi@@94
 \mathchardef\bmupdelta="0\mtbmi@@95
 \mathchardef\bmupepsilon="0\mtbmi@@96
 \mathchardef\bmupzeta="0\mtbmi@@97
 \mathchardef\bmupeta="0\mtbmi@@98
 \mathchardef\bmuptheta="0\mtbmi@@99
 \mathchardef\bmupiota="0\mtbmi@@9A
 \mathchardef\bmupkappa="0\mtbmi@@9B
 \mathchardef\bmuplambda="0\mtbmi@@9C
 \mathchardef\bmupmu="0\mtbmi@@9D
 \mathchardef\bmupnu="0\mtbmi@@9E
 \mathchardef\bmupxi="0\mtbmi@@9F
 \mathchardef\bmuppi="0\mtbmi@@ A0
 \mathchardef\bmuprho="0\mtbmi@@ A1
 \mathchardef\bmupsigma="0\mtbmi@@ A2
 \mathchardef\bmuptau="0\mtbmi@@ A3
 \mathchardef\bmupupsilon="0\mtbmi@@ A4
 \mathchardef\bmupphi="0\mtbmi@@ A5
 \mathchardef\bmupchi="0\mtbmi@@ A6
 \mathchardef\bmuppsi="0\mtbmi@@ A7
 \mathchardef\bmupomega="0\mtbmi@@ A8
 \mathchardef\bmupvarepsilon="0\mtbmi@@ A9
 \mathchardef\bmupvartheta="0\mtbmi@@ AA
 \mathchardef\bmupvarpi="0\mtbmi@@ AB
 \mathchardef\bmupvarrho="0\mtbmi@@ AC
 \mathchardef\bmupvarsigma="0\mtbmi@@ AD
 \mathchardef\bmupvarphi="0\mtbmi@@ AE
 \mathchardef\bmupvarkappa="0\mtbmi@@ AF
 \mathchardef\bmcdot="2\mtbsy@@01
 \mathchardef\bmtimes="2\mtbsy@@02
 \mathchardef\bmast="2\mtbsy@@03
 \mathchardef\bmdiv="2\mtbsy@@04
 \mathchardef\bmdiamond="2\mtbsy@@05
 \mathchardef\bmpm="2\mtbsy@@06
 \mathchardef\bmmp="2\mtbsy@@07
 \mathchardef\bmoplus="2\mtbsy@@08
 \mathchardef\bmominus="2\mtbsy@@09
 \mathchardef\bmotimes="2\mtbsy@@0A
 \mathchardef\bmoslash="2\mtbsy@@0B
 \mathchardef\bmodot="2\mtbsy@@0C
 \mathchardef\bmbigcirc="2\mtbsy@@0D
 \mathchardef\bmcirc="2\mtbsy@@0E
 \mathchardef\bmbullet="2\mtbsy@@0F
 \mathchardef\bmasymp="3\mtbsy@@10
 \mathchardef\bmequiv="3\mtbsy@@11
 \mathchardef\bmsubseteq="3\mtbsy@@12
 \mathchardef\bmsupseteq="3\mtbsy@@13
 \mathchardef\bmleq="3\mtbsy@@14 \let\bmle=\bmleq
 \mathchardef\bmgeq="3\mtbsy@@15 \let\bmge=\bmgeq
 \mathchardef\bmpreceq="3\mtbsy@@16
 \mathchardef\bmsucceq="3\mtbsy@@17
 \mathchardef\bmsim="3\mtbsy@@18
 \mathchardef\bmapprox="3\mtbsy@@19
 \mathchardef\bmsubset="3\mtbsy@@1A
 \mathchardef\bmsupset="3\mtbsy@@1B
 \mathchardef\bmll="3\mtbsy@@1C
 \mathchardef\bmgg="3\mtbsy@@1D
 \mathchardef\bmprec="3\mtbsy@@1E
 \mathchardef\bmsucc="3\mtbsy@@1F
 \mathchardef\bmleftarrow="3\mtbsy@@20 \let\bmgets=\bmleftarrow
 \mathchardef\bmrightarrow="3\mtbsy@@21 \let\bmto=\bmrightarrow
 \def\bmuparrow{\delimiter"3\mtbsyt@@22\mtbex@@78 }%
 \def\bmdownarrow{\delimiter"3\mtbsyt@@23\mtbex@@79 }%
 \mathchardef\bmleftrightarrow="3\mtbsy@@24
 \def\bmuparrow{\delimiter"3\mtbsy@@22378 }%
 \def\bmdownarrow{\delimiter"3\mtbsy@@23379 }%
 \mathchardef\bmnearrow="3\mtbsy@@25
 \mathchardef\bmsearrow="3\mtbsy@@26
 \mathchardef\bmsimeq="3\mtbsy@@27
 \mathchardef\bmLeftarrow="3\mtbsy@@28
 \mathchardef\bmRightarrow="3\mtbsy@@29
 \def\bmUparrow{\delimiter"3\mtbsy@@2A\mtbex@@7E }%
 \def\bmDownarrow{\delimiter"3\mtbsy@@2B\mtbex@@7F }%
 \mathchardef\bmLeftrightarrow="3\mtbsy@@2C
 \mathchardef\bmnwarrow="3\mtbsy@@2D
 \mathchardef\bmswarrow="3\mtbsy@@2E
 \mathchardef\bmpropto="3\mtbsy@@2F
 \mathchardef\bmprime="0\mtbsy@@30
 \mathchardef\bminfty="0\mtbsy@@31
 \mathchardef\bmin="3\mtbsy@@32
 \mathchardef\bmni="3\mtbsy@@33 \let\bmowns=\bmni
 \mathchardef\bmbigtriangleup="2\mtbsy@@34
 \mathchardef\bmtriangle="0\mtbsy@@34
 \mathchardef\bmbigtriangledown="2\mtbsy@@35
 \mathchardef\bmforall="0\mtbsy@@38
 \mathchardef\bmexists="0\mtbsy@@39
 \mathchardef\bmneg="0\mtbsy@@3A \let\lnot=\neg
 \mathchardef\bmemptyset="0\mtbsy@@3B
 \mathchardef\bmRe="0\mtbsy@@3C
 \mathchardef\bmIm="0\mtbsy@@3D
 \mathchardef\bmtop="0\mtbsy@@3E
 \mathchardef\bmbot="0\mtbsy@@3F
 \mathchardef\bmperp="3\mtbsy@@3F
 \mathchardef\bmaleph="0\mtbsy@@40
 \mathchardef\bmcomp="2\mtbsy@@42
 \def\bmvec{\mathaccent"0\mtbsy@@45 }%
 \mathchardef\bmtriangleright="2\mtbsy@@46
 \mathchardef\bmtriangleleft="2\mtbsy@@47
 \mathchardef\bmcolon="6\mtbsy@@57
 \mathchardef\bmsetdif="2\mtbsy@@58
 \mathchardef\bmcupprod="2\mtbsy@@59
 \mathchardef\bmcapprod="2\mtbsy@@5A
 \mathchardef\bmcup="2\mtbsy@@5B
 \mathchardef\bmcap="2\mtbsy@@5C
 \mathchardef\bmuplus="2\mtbsy@@5D
 \mathchardef\bmwedge="2\mtbsy@@5E \let\bmland=\bmwedge
 \mathchardef\bmvee="2\mtbsy@@5F \let\bmlor=\bmvee
 \mathchardef\bmvdash="3\mtbsy@@60
 \mathchardef\bmdashv="3\mtbsy@@61
 \def\bmlfloor{\delimiter"4\mtbsy@@62\mtbex@@04 }%
 \def\bmrfloor{\delimiter"5\mtbsy@@63\mtbex@@05 }%
 \def\bmlceil{\delimiter"4\mtbsy@@64\mtbex@@06 }%
 \def\bmrceil{\delimiter"5\mtbsy@@65\mtbex@@07 }%
 \def\bmlbrace{\delimiter"4\mtbsy@@66\mtbex@@08 }% 
 \def\bmrbrace{\delimiter"5\mtbsy@@67\mtbex@@09 }%
 \def\bmlangle{\delimiter"4\mtbsy@@68\mtbex@@0A }%
 \def\bmrangle{\delimiter"5\mtbsy@@69\mtbex@@0B }%
 \mathchardef\bmmid="3\mtbsy@@6A
 \def\bmvert{\delimiter"\mtbsy@@6A\mtbex@@0C }%
 \mathchardef\bmparallel="3\mtbsy@@6B
 \def\bmVert{\delimiter"\mtbsy@@6B\mtbex@@0D }%
 \def\bmupdownarrow{\delimiter"3\mtbsy@@6C\mtbex@@3F }%
 \def\bmUpdownarrow{\delimiter"3\mtbsy@@6D\mtbex@@77 }%
 \def\bmbackslash{\delimiter"\mtbsy@@6E\mtbex@@0F }%
 \def\bmarrowvert{\delimiter"\mtbsy@@6A\mtbex@@3C }%
 \def\bmArrowvert{\delimiter"\mtbsy@@6\mtbex@@3D }%
 \def\bmlgroup{\delimiter"4\mtbmi@@2E\mtbex@@3A }% 
 \def\bmrgroup{\delimiter"5\mtbmi@@2F\mtbex@@3B }% 
 \def\bmbracevert{\delimiter"\mtbex@@8D\mtbex@@8D }% 
 \mathchardef\bmsetminus="2\mtbsy@@6E 
 \mathchardef\bmwr="2\mtbsy@@6F
 \def\bmsurd{{\mathchar"1\mtbsy@@70}}%
 \mathchardef\bmamalg="2\mtbsy@@71
 \mathchardef\bmnabla="0\mtbsy@@72
 \mathchardef\bmsmallint="1\mtbsy@@73
 \mathchardef\bmsqcup="2\mtbsy@@74
 \mathchardef\bmsqcap="2\mtbsy@@75
 \mathchardef\bmsqsubseteq="3\mtbsy@@76
 \mathchardef\bmsqsupseteq="3\mtbsy@@77
 \mathchardef\bmclubsuit="0\mtbsy@@7C
 \mathchardef\bmdiamondsuit="0\mtbsy@@7D
 \mathchardef\bmheartsuit="0\mtbsy@@7E
 \mathchardef\bmspadesuit="0\mtbsy@@7F
 \mathchardef\bmhbar="0\mtbsy@@84
 \mathchardef\bmnotin="3\mtbsy@@85
 \mathchardef\bmangle="0\mtbsy@@86
 \mathchardef\bmdoteq="3\mtbsy@@87
 \mathchardef\bmmodels="3\mtbsy@@88
 \mathchardef\bmbowtie="3\mtbsy@@89
 \mathchardef\bmcong="3\mtbsy@@8A
 \mathchardef\bmhookleftarrow="3\mtbsy@@8B
 \mathchardef\bmhookrightarrow="3\mtbsy@@8C
 \mathchardef\bmlongleftarrow="3\mtbsy@@8D
 \mathchardef\bmlongrightarrow="3\mtbsy@@8E
 \mathchardef\bmLongleftarrow="3\mtbsy@@8F
 \mathchardef\bmLongrightarrow="3\mtbsy@@90
 \mathchardef\bmmapsto="3\mtbsy@@91
 \mathchardef\bmlongmapsto="3\mtbsy@@92
 \mathchardef\bmlongleftrightarrow="3\mtbsy@@93
 \mathchardef\bmLongleftrightarrow="3\mtbsy@@94
 \def\bmiff{\;\bmLongleftrightarrow\;}%
 \mathchardef\bmrightleftharpoons="3\mtbsy@@95
 \mathchardef\bmnotless="3\mtbsy@@96
 \mathchardef\bmnotleq="3\mtbsy@@97
 \mathchardef\bmnotprec="3\mtbsy@@98
 \mathchardef\bmnotpreceq="3\mtbsy@@99
 \mathchardef\bmnotsubset="3\mtbsy@@9A
 \mathchardef\bmnotsubseteq="3\mtbsy@@9B
 \mathchardef\bmnotsqsubseteq="3\mtbsy@@9C
 \mathchardef\bmnotgr="3\mtbsy@@9D
 \mathchardef\bmnotgeq="3\mtbsy@@9E
 \mathchardef\bmnotsucc="3\mtbsy@@9F
 \mathchardef\bmnotsucceq="3\mtbsy@@ A0
 \mathchardef\bmnotsupset="3\mtbsy@@ A1
 \mathchardef\bmnotsupseteq="3\mtbsy@@ A2
 \mathchardef\bmnotsqsupseteq="3\mtbsy@@ A3
 \mathchardef\bmneq="3\mtbsy@@ A4 \let\bmne=\bmneq
 \mathchardef\bmnotequiv="3\mtbsy@@ A5
 \mathchardef\bmnotsim="3\mtbsy@@ A6
 \mathchardef\bmnotsimeq="3\mtbsy@@ A7
 \mathchardef\bmnotapprox="3\mtbsy@@ A8
 \mathchardef\bmnotcong="3\mtbsy@@ A9
 \mathchardef\bmnotasymp="3\mtbsy@@ AA
 \def\bmgrave{\ifnum\fam=\m@ne\mathaccent"0\mtbsy@@4A \else\@grave@\fi}%
 \def\bmacute{\ifnum\fam=\m@ne\mathaccent"0\mtbsy@@4B \else\@acute@\fi}%
 \def\bmcheck{\ifnum\fam=\m@ne\mathaccent"0\mtbsy@@4C \else\@check@\fi}%
 \def\bmbreve{\ifnum\fam=\m@ne\mathaccent"0\mtbsy@@4D \else\@breve@\fi}%
 \def\bmbar{\ifnum\fam=\m@ne\mathaccent"0\mtbsy@@4E \else\@bar@\fi}%
 \def\bmhat{\ifnum\fam=\m@ne\mathaccent"0\mtbsy@@4F \else\@hat@\fi}%
 \def\bmdot{\ifnum\fam=\m@ne\mathaccent"0\mtbsy@@50 \else\@dot@\fi}%
 \def\bmtilde{\ifnum\fam=\m@ne\mathaccent"0\mtbsy@@51 \else\@tilde@\fi}%
 \def\bmddot{\ifnum\fam=\m@ne\mathaccent"0\mtbsy@@52 \else\@ddot@\fi}%
 \def\bmdotup{\mathaccent"0\mtbsy@@54 }%
 \def\bmddotup{\mathaccent"0\mtbsy@@55 }%
 \def\bmoacc{\mathaccent"0\mtbsy@@56 }%
 \def\bmdddot{\mathaccent"0\mtbsy@@ AB }%
 \def\bmddddot{\mathaccent"0\mtbsy@@ AC }%
 \def\bmdddotup{\mathaccent"0\mtbsy@@ AD}%
 \def\bmddddotup{\mathaccent"0\mtbsy@@ AE }%
 \mathchardef\bmhslash="\mtbsy@@ AF
 \mathchardef\bmdigamma="0\mtbsy@@ B1
 \def\bmyen{\ifmmode\mathchar"0\mtbsy@@ B0 \else\mathhexbox\mtbsy@@ B0\fi}%
 \ifx\amstexloaded@\relax
  \Makeacc@\bmHat{\mtbsy@@4F}%
  \Makeacc@\bmCheck{\mtbsy@@4C}%
  \Makeacc@\bmTilde{\mtbsy@@51}%
  \Makeacc@\bmAcute{\mtbsy@@4B}%
  \Makeacc@\bmGrave{\mtbsy@@4A}%
  \Makeacc@\bmDot{\mtbsy@@50}%
  \Makeacc@\bmDdot{\mtbsy@@52}%
  \Makeacc@\bmDotup{\mtbsy@@54}%
  \Makeacc@\bmDdotup{\mtbsy@@55}%
  \Makeacc@\bmBreve{\mtbsy@@4D}%
  \Makeacc@\bmBar{\mtbsy@@4E}%
  \Makeacc@\bmOacc{\mtbsy@@56}%
  \Makeacc@\bmDddot{\mtbsy@@ AB}%
  \Makeacc@\bmDdddot{\mtbsy@@ AC}%
  \Makeacc@\bmDddotup{\mtbsy@@ AD}%
  \Makeacc@\bmDdddotup{\mtbsy@@ AE}%
 \fi
 \def\bmwbar{\mathaccent"0\mtbsy@@78 }%
 \def\bmwwbar{\mathaccent"0\mtbsy@@53 }%
 \def\bmwhat{\mathaccent"0\mtbsy@@79 }%
 \def\bmwtilde{\mathaccent"0\mtbsy@@7A }%
 \def\bmwcheck{\mathaccent"0\mtbsy@@7B }%
 \def\bmwwhat{\mathaccent"0\mtbex@@80 }%
 \def\bmwwtilde{\mathaccent"0\mtbex@@81 }%
 \def\bmwwcheck{\mathaccent"0\mtbex@@7D }%
 \def\bmwidehat{\mathaccent"0\mtbex@@62 }%
 \def\bmwidetilde{\mathaccent"0\mtbex@@65 }%
 \def\bmwidecheck{\mathaccent"0\mtbex@@7A }%
 \ifx\amstexloaded@\relax
  \Makeacc@\bmWbar{\mtbsy@@78 }%
  \Makeacc@\bmWwbar{\mtbsy@@53 }%
  \Makeacc@\bmWhat{\mtbsy@@79 }%
  \Makeacc@\bmWtilde{\mtbsy@@7A }%
  \Makeacc@\bmWcheck{\mtbsy@@7B }%
  \Makeacc@\bmWwhat{\mtbex@@80 }%
  \Makeacc@\bmWwtilde{\mtbex@@81 }%
  \Makeacc@\bmWwcheck{\mtbex@@7D }%
 \fi
 \mathchardef\bmcoprod="1\mtbex@@60
 \mathchardef\bmbigvee="1\mtbex@@57
 \mathchardef\bmbigwedge="1\mtbex@@56
 \mathchardef\bmbiguplus="1\mtbex@@55
 \mathchardef\bmbigcap="1\mtbex@@54
 \mathchardef\bmbigcup="1\mtbex@@53
 \mathchardef\bmintop="1\mtbex@@52 \def\bmint{\bmintop\nolimits}%
 \mathchardef\bmprod="1\mtbex@@51
 \mathchardef\bmsum="1\mtbex@@50
 \mathchardef\bmbigotimes="1\mtbex@@4E
 \mathchardef\bmbigoplus="1\mtbex@@4C
 \mathchardef\bmbigodot="1\mtbex@@4A
 \mathchardef\bmointop="1\mtbex@@48 \def\bmoint{\ointop\nolimits}%
 \mathchardef\bmiintop="1\mtbex@@92 \def\bmiint{\bmiintop\nolimits}%
 \mathchardef\bmiiintop="1\mtbex@@94 \def\bmiiint{\bmiiintop\nolimits}%
 \mathchardef\bmoiintop="1\mtbex@@96 \def\bmoiint{\bmoiintop\nolimits}%
 \mathchardef\bmoiiintop="1\mtbex@@98 \def\bmoiiint{\bmoiiintop\nolimits}%
 \mathchardef\bmcwointop="1\mtbex@@9A \def\bmcwoint{\bmcwointop\nolimits}%
 \mathchardef\bmawointop="1\mtbex@@9C \def\bmawoint{\bmawointop\nolimits}%
 \mathchardef\bmcwintop="1\mtbex@@9E \def\bmcwint{\bmcwintop\nolimits}%
 \mathchardef\bmbigsqcup="1\mtbex@@46
 \def\bmlmoustache{\delimiter"4\mtbex@@7A\mtbex@@40 }%
 \def\bmrmoustahce{\delimiter"5\mtbex@@7B\mtbex@@41 }% 
 \def\bmlgroup{\delimiter"4\mtbmi@@2E\mtbex@@3A }% 
 \def\bmrgroup{\delimiter"5\mtbmi@@2F\mtbex@@3B }%
 \def\bmbracevert{\delimiter"\mtbex@@3E\mtbex@@3E }%
 \def\bmchoose{\atopwithdelims\bmlparens\bmrparens}%
 \def\bmbrack{\atopwithdelims\bmlbrack\bmrbrack}%
 \def\bmbrace{\atopwithdelims\bmlbrace\bmrbrace}%
}



\def\boldmath{%
 \textfont\z@\the\textfont\bffam
 \scriptfont\z@\the\scriptfont\bffam
 \scriptscriptfont\z@\the\scriptscriptfont\bffam
 \textfont\@ne\the\textfont\mtbmi@
 \scriptfont\@ne\the\scriptfont\mtbmi@
 \scriptscriptfont\@ne\the\scriptscriptfont\mtbmi@
 \textfont\tw@\the\textfont\mtbsy@
 \scriptfont\tw@\the\scriptfont\mtbsy@
 \scriptscriptfont\tw@\the\scriptscriptfont\mtbsy@
 \textfont\thr@@\the\textfont\mtbex@
 \scriptfont\thr@@\the\scriptfont\mtbex@
 \scriptscriptfont\thr@@\the\scriptscriptfont\mtbex@
 \ifx\p@renwd\undefined
  \else
  \setbox\z@\hbox{\the\textfont\mtbex@ B}\p@renwd\wd\z@
 \fi
 \ifx\amstexloaded@\relax
  \buffer@\fontdimen13\the\textfont\mtbex@
  \buffer\buffer@
 \fi
 \let\lmoustache\bmlmoustache % these 5 delimiters have different definitions for mtexa !
 \let\rmoustache\bmrmoustahce 
 \let\lgroup\bmlgroup
 \let\rgroup\bmrgroup
 \let\bracevert\bmbracevert
 \let\SQRT\sqrt
 \def\ROOT##1\OF##2{\root##1\of{##2}}%
 \def\PARENS##1{\left(##1\right)}%
 \def\LEFTRIGHT##1##2##3{\left##1##2\right##3}%
 \let\widehat\bmwidehat % these 3 wide accents have different definitions for mtexa !
 \let\widetilde\bmwidetilde
 \let\widecheck\bmwidecheck
}

% Now come the macros for the heavy fonts.

\def\hm{\futurelet\next\hm@}
\def\hm@{\ifx\next'\def\next@##1{\hprime@}\else
 \ifcat\noexpand\next0\def\next@{\hm@@}\else
 \def\next@{\errmessage{Invalid use of \string\hm}}\fi\fi
 \next@}
\newcount\firsthprimesep
\firsthprimesep=2
\def\hprime@{^\bgroup\firstprime@true\hprim@s}
\def\hprim@s{\iffirstprime@\kern\firsthprimesep\emcent@\fi
  \hmprime\firstprime@false\futurelet\next\hpr@m@s}
\newcount\nonhprimesep
\nonhprimesep=7
\def\hpr@m@s{\ifx'\next\def\next@##1{\hprim@s}\else
 \ifx^\next\def\next@##1##2{\kern\nonhprimesep\emcent@##2\egroup}\else
 \let\next@\egroup\fi\fi\next@}
\def\hm@@#1{%
  \ifx#1+\mathchar"2\mthsy@@43
  \else\ifx#1-\mathchar"2\mthsy@@00
  \else\ifx#1=\mathchar"3\mthsy@@44
  \else\ifx#1<\mathchar"3\mthsy@@ EA
  \else\ifx#1>\mathchar"3\mthsy@@ EC
  \else\ifx#1/\mathchar"0\mthsy@@ EB
  \else\ifx#1(\mathchar"4\mthsy@@ E6
  \else\ifx#1)\mathchar"5\mthsy@@ E7
  \else\ifx#1[\mathchar"4\mthsy@@ F8
  \else\ifx#1]\mathchar"5\mthsy@@ F9
  \else\ifx#1|\mathchar"0\mthsy@@6A
  \else\ifx#1*\mathchar"2\mthsy@@03
  \else\ifx#1.\mathchar"0\mthsy@@ E8
  \else\ifx#1,\mathchar"6\mthsy@@ E9
  \else\ifx#1;\mathchar"6\mthsy@@49 
  \else\ifx#1:\mathchar"3\mthsy@@57
  \else\ifx#1!\mathchar"5\mthsy@@ F6
  \else\ifx#1?\mathchar"5\mthsy@@ F7
  \else\errmessage{Invalid use of \string\hm}%
  \fi\fi\fi\fi\fi\fi\fi\fi\fi\fi\fi\fi\fi\fi\fi\fi\fi\fi}
\def\hmdefs@{%
 \mathchardef\hmleftharpoonup="3\mthsy@@ E0
 \mathchardef\hmleftharpoondown="3\mthsy@@ E1
 \mathchardef\hmrightharpoonup="3\mthsy@@ E2
 \mathchardef\hmrightharpoondown="3\mthsy@@ E3
 \def\hmlparens{\delimiter"\mthsy@@ E6\mthex@@00 }%
 \def\hmrparens{\delimiter"\mthsy@@ E7\mthex@@01 }%
 \def\hmslash{\delimiter"\mthsy@@ EB\mthex@@0E }%
 \mathchardef\hmstar="2\mthsy@@ ED
 \mathchardef\hmpartial="0\mthsy@@ EE
 \mathchardef\hmflat="0\mthsy@@ EF
 \mathchardef\hmnatural="0\mthsy@@ F0
 \mathchardef\hmsharp="0\mthsy@@ F1
 \mathchardef\hmsmile="3\mthsy@@ F2
 \mathchardef\hmfrown="3\mthsy@@ F3
 \mathchardef\hmell="0\mthsy@@ F4
 \mathchardef\hmwp="0\mthsy@@ F5
 \def\hmlbrack{\delimiter"4\mthsy@@ F8\mthex@@02 }%
 \def\hmrbrack{\delimiter"5\mthsy@@ F9\mthex@@03 }%
 \mathchardef\hmdagger="2\mthsy@@ FA
 \mathchardef\hmddagger="2\mthsy@@ FB
 \mathchardef\hmcdot="2\mthsy@@01
 \mathchardef\hmtimes="2\mthsy@@02
 \mathchardef\hmast="2\mthsy@@03
 \mathchardef\hmdiv="2\mthsy@@04
 \mathchardef\hmdiamond="2\mthsy@@05
 \mathchardef\hmpm="2\mthsy@@06
 \mathchardef\hmmp="2\mthsy@@07
 \mathchardef\hmoplus="2\mthsy@@08
 \mathchardef\hmominus="2\mthsy@@09
 \mathchardef\hmotimes="2\mthsy@@0A
 \mathchardef\hmoslash="2\mthsy@@0B
 \mathchardef\hmodot="2\mthsy@@0C
 \mathchardef\hmbigcirc="2\mthsy@@0D
 \mathchardef\hmcirc="2\mthsy@@0E
 \mathchardef\hmbullet="2\mthsy@@0F
 \mathchardef\hmasymp="3\mthsy@@10
 \mathchardef\hmequiv="3\mthsy@@11
 \mathchardef\hmsubseteq="3\mthsy@@12
 \mathchardef\hmsupseteq="3\mthsy@@13
 \mathchardef\hmleq="3\mthsy@@14 \let\hmle=\hmleq
 \mathchardef\hmgeq="3\mthsy@@15 \let\hmge=\hmgeq
 \mathchardef\hmpreceq="3\mthsy@@16
 \mathchardef\hmsucceq="3\mthsy@@17
 \mathchardef\hmsim="3\mthsy@@18
 \mathchardef\hmapprox="3\mthsy@@19
 \mathchardef\hmsubset="3\mthsy@@1A
 \mathchardef\hmsupset="3\mthsy@@1B
 \mathchardef\hmll="3\mthsy@@1C
 \mathchardef\hmgg="3\mthsy@@1D
 \mathchardef\hmprec="3\mthsy@@1E
 \mathchardef\hmsucc="3\mthsy@@1F
 \mathchardef\hmleftarrow="3\mthsy@@20 \let\hmgets=\hmleftarrow
 \mathchardef\hmrightarrow="3\mthsy@@21 \let\hmto=\hmrightarrow
 \def\hmuparrow{\delimiter"3\mthsy@@22\mthex@@78 }%
 \def\hmdownarrow{\delimiter"3\mthsy@@23\mthex@@79 }%
 \mathchardef\hmleftrightarrow="3\mthsy@@24
 \def\hmuparrow{\delimiter"3\mthsy@@22378 }%
 \def\hmdownarrow{\delimiter"3\mthsy@@23379 }%
 \mathchardef\hmnearrow="3\mthsy@@25
 \mathchardef\hmsearrow="3\mthsy@@26
 \mathchardef\hmsimeq="3\mthsy@@27
 \mathchardef\hmLeftarrow="3\mthsy@@28
 \mathchardef\hmRightarrow="3\mthsy@@29
 \def\hmUparrow{\delimiter"3\mthsy@@2A\mthex@@7E }%
 \def\hmDownarrow{\delimiter"3\mthsy@@2B\mthex@@7F }%
 \mathchardef\hmLeftrightarrow="3\mthsy@@2C
 \mathchardef\hmnwarrow="3\mthsy@@2D
 \mathchardef\hmswarrow="3\mthsy@@2E
 \mathchardef\hmpropto="3\mthsy@@2F
 \mathchardef\hmprime="0\mthsy@@30
 \mathchardef\hminfty="0\mthsy@@31
 \mathchardef\hmin="3\mthsy@@32
 \mathchardef\hmni="3\mthsy@@33 \let\hmowns=\hmni
 \mathchardef\hmbigtriangleup="2\mthsy@@34
 \mathchardef\hmtriangle="0\mthsy@@34
 \mathchardef\hmbigtriangledown="2\mthsy@@35
 \mathchardef\hmnot="3\mthsy@@36
 \mathchardef\hmmapstochar="3\mthsy@@37 
 \mathchardef\hmforall="0\mthsy@@38
 \mathchardef\hmexists="0\mthsy@@39
 \mathchardef\hmneg="0\mthsy@@3A \let\lnot=\neg
 \mathchardef\hmemptyset="0\mthsy@@3B
 \mathchardef\hmRe="0\mthsy@@3C
 \mathchardef\hmIm="0\mthsy@@3D
 \mathchardef\hmtop="0\mthsy@@3E
 \mathchardef\hmbot="0\mthsy@@3F
 \mathchardef\hmperp="3\mthsy@@3F
 \mathchardef\hmaleph="0\mthsy@@40
 \mathchardef\hmcomp="2\mthsy@@42
 \def\hmvec{\mathaccent"0\mthsy@@45 }%
 \mathchardef\hmtriangleright="2\mthsy@@46
 \mathchardef\hmtriangleleft="2\mthsy@@47
 \mathchardef\hmcolon="6\mthsy@@57
 \mathchardef\hmsetdif="2\mthsy@@58
 \mathchardef\hmcupprod="2\mthsy@@59
 \mathchardef\hmcapprod="2\mthsy@@5A
 \mathchardef\hmcup="2\mthsy@@5B
 \mathchardef\hmcap="2\mthsy@@5C
 \mathchardef\hmuplus="2\mthsy@@5D
 \mathchardef\hmwedge="2\mthsy@@5E \let\hmland=\hmwedge
 \mathchardef\hmvee="2\mthsy@@5F \let\hmlor=\hmvee
 \mathchardef\hmvdash="3\mthsy@@60
 \mathchardef\hmdashv="3\mthsy@@61
 \def\hmlfloor{\delimiter"4\mthsy@@62\mthex@@04 }%
 \def\hmrfloor{\delimiter"5\mthsy@@63\mthex@@05 }%
 \def\hmlceil{\delimiter"4\mthsy@@64\mthex@@06 }%
 \def\hmrceil{\delimiter"5\mthsy@@65\mthex@@07 }%
 \def\hmlbrace{\delimiter"4\mthsy@@66\mthex@@08 }% 
 \def\hmrbrace{\delimiter"5\mthsy@@67\mthex@@09 }%
 \def\hmlangle{\delimiter"4\mthsy@@68\mthex@@0A }%
 \def\hmrangle{\delimiter"5\mthsy@@69\mthex@@0B }%
 \mathchardef\hmmid="3\mthsy@@6A
 \def\hmvert{\delimiter"\mthsy@@6A\mthex@@0C }%
 \mathchardef\hmparallel="3\mthsy@@6B
 \def\hmVert{\delimiter"\mthsy@@6B\mthex@@0D }%
 \def\hmupdownarrow{\delimiter"3\mthsy@@6C\mthex@@3F }%
 \def\hmUpdownarrow{\delimiter"3\mthsy@@6D\mthex@@77 }%
 \def\hmbackslash{\delimiter"\mthsy@@6E\mthex@@0F }%
 \def\hmarrowvert{\delimiter"\mthsy@@6A\mthex@@3C }%
 \def\hmArrowvert{\delimiter"\mthsy@@6\mthex@@3D }%
 \def\hmlgroup{\delimiter"4\mthsy@@2E\mthex@@3A }% 
 \def\hmrgroup{\delimiter"5\mthsy@@2F\mthex@@3B }% 
 \def\hmbracevert{\delimiter"\mthex@@8D\mthex@@8D }% 
 \mathchardef\hmsetminus="2\mthsy@@6E 
 \mathchardef\hmwr="2\mthsy@@6F
 \def\hmsurd{{\mathchar"1\mthsy@@70}}%
 \mathchardef\hmamalg="2\mthsy@@71
 \mathchardef\hmnabla="0\mthsy@@72
 \mathchardef\hmsmallint="1\mthsy@@73
 \mathchardef\hmsqcup="2\mthsy@@74
 \mathchardef\hmsqcap="2\mthsy@@75
 \mathchardef\hmsqsubseteq="3\mthsy@@76
 \mathchardef\hmsqsupseteq="3\mthsy@@77
 \mathchardef\hmclubsuit="0\mthsy@@7C
 \mathchardef\hmdiamondsuit="0\mthsy@@7D
 \mathchardef\hmheartsuit="0\mthsy@@7E
 \mathchardef\hmspadesuit="0\mthsy@@7F
 \mathchardef\hmnotin="3\mthsy@@85
 \mathchardef\hmangle="0\mthsy@@86
 \mathchardef\hmdoteq="3\mthsy@@87
 \mathchardef\hmmodels="3\mthsy@@88
 \mathchardef\hmbowtie="3\mthsy@@89
 \mathchardef\hmcong="3\mthsy@@8A
 \mathchardef\hmhookleftarrow="3\mthsy@@8B
 \mathchardef\hmhookrightarrow="3\mthsy@@8C
 \mathchardef\hmlongleftarrow="3\mthsy@@8D
 \mathchardef\hmlongrightarrow="3\mthsy@@8E
 \mathchardef\hmLongleftarrow="3\mthsy@@8F
 \mathchardef\hmLongrightarrow="3\mthsy@@90
 \mathchardef\hmmapsto="3\mthsy@@91
 \mathchardef\hmlongmapsto="3\mthsy@@92
 \mathchardef\hmlongleftrightarrow="3\mthsy@@93
 \mathchardef\hmLongleftrightarrow="3\mthsy@@94
 \def\hmiff{\;\hmLongleftrightarrow\;}%
 \mathchardef\hmrightleftharpoons="3\mthsy@@95
 \mathchardef\hmnotless="3\mthsy@@96
 \mathchardef\hmnotleq="3\mthsy@@97
 \mathchardef\hmnotprec="3\mthsy@@98
 \mathchardef\hmnotpreceq="3\mthsy@@99
 \mathchardef\hmnotsubset="3\mthsy@@9A
 \mathchardef\hmnotsubseteq="3\mthsy@@9B
 \mathchardef\hmnotsqsubseteq="3\mthsy@@9C
 \mathchardef\hmnotgr="3\mthsy@@9D
 \mathchardef\hmnotgeq="3\mthsy@@9E
 \mathchardef\hmnotsucc="3\mthsy@@9F
 \mathchardef\hmnotsucceq="3\mthsy@@ A0
 \mathchardef\hmnotsupset="3\mthsy@@ A1
 \mathchardef\hmnotsupseteq="3\mthsy@@ A2
 \mathchardef\hmnotsqsupseteq="3\mthsy@@ A3
 \mathchardef\hmneq="3\mthsy@@ A4 \let\hmne=\hmneq
 \mathchardef\hmnotequiv="3\mthsy@@ A5
 \mathchardef\hmnotsim="3\mthsy@@ A6
 \mathchardef\hmnotsimeq="3\mthsy@@ A7
 \mathchardef\hmnotapprox="3\mthsy@@ A8
 \mathchardef\hmnotcong="3\mthsy@@ A9
 \mathchardef\hmnotasymp="3\mthsy@@ AA
 \mathchardef\hmangle="2\mthsy@@86
 \mathchardef\hmdigamma="0\mthsy@@ B1
 \def\hmyen{\ifmmode\mathchar"0\mthsy@@ B0 \else\mathhexbox\mthsy@@ B0\fi}%
 \def\hmgrave{\ifnum\fam=\m@ne\mathaccent"0\mthsy@@4A \else\@grave@\fi}%
 \def\hmacute{\ifnum\fam=\m@ne\mathaccent"0\mthsy@@4B \else\@acute@\fi}%
 \def\hmcheck{\ifnum\fam=\m@ne\mathaccent"0\mthsy@@4C \else\@check@\fi}%
 \def\hmbreve{\ifnum\fam=\m@ne\mathaccent"0\mthsy@@4D \else\@breve@\fi}%
 \def\hmbar{\ifnum\fam=\m@ne\mathaccent"0\mthsy@@4E \else\@bar@\fi}%
 \def\hmhat{\ifnum\fam=\m@ne\mathaccent"0\mthsy@@4F \else\@hat@\fi}%
 \def\hmdot{\ifnum\fam=\m@ne\mathaccent"0\mthsy@@50 \else\@dot@\fi}%
 \def\hmtilde{\ifnum\fam=\m@ne\mathaccent"0\mthsy@@51 \else\@tilde@\fi}%
 \def\hmddot{\ifnum\fam=\m@ne\mathaccent"0\mthsy@@52 \else\@ddot@\fi}%
 \def\hmdotup{\mathaccent"0\mthsy@@54 }%
 \def\hmddotup{\mathaccent"0\mthsy@@55 }%
 \def\hmoacc{\mathaccent"0\mthsy@@56 }%
 \def\hmdddot{\mathaccent"0\mthsy@@ AB }%
 \def\hmddddot{\mathaccent"0\mthsy@@ AC }%
 \def\hmdddotup{\mathaccent"0\mthsy@@ AD }%
 \def\hmddddotup{\mathaccent"0\mthsy@@ AE }%
 \ifx\amstexloaded@\relax
  \Makeacc@\hmHat{\mthsy@@4F}%
  \Makeacc@\hmCheck{\mthsy@@4C}%
  \Makeacc@\hmTilde{\mthsy@@51}%
  \Makeacc@\hmAcute{\mthsy@@4B}%
  \Makeacc@\hmGrave{\mthsy@@4A}%
  \Makeacc@\hmDot{\mthsy@@50}%
  \Makeacc@\hmDdot{\mthsy@@52}%
  \Makeacc@\hmDotup{\mthsy@@54}%
  \Makeacc@\hmDdotup{\mthsy@@55}%
  \Makeacc@\hmBreve{\mthsy@@4D}%
  \Makeacc@\hmBar{\mthsy@@4E}%
  \Makeacc@\hmOacc{\mthsy@@56}%
  \Makeacc@\hmDddot{\mthsy@@ AB}%
  \Makeacc@\hmDdddot{\mthsy@@ AC}%
  \Makeacc@\hmDddotup{\mthsy@@ AD}%
  \Makeacc@\hmDdddotup{\mthsy@@ AE}% 
 \fi
 \def\hmwbar{\mathaccent"0\mthsy@@78 }%
 \def\hmwwbar{\mathaccent"0\mthsy@@53 }%
 \def\hmwhat{\mathaccent"0\mthsy@@79 }%
 \def\hmwtilde{\mathaccent"0\mthsy@@7A }%
 \def\hmwcheck{\mathaccent"0\mthsy@@7B }%
 \def\hmwwhat{\mathaccent"0\mthex@@80 }%
 \def\hmwwtilde{\mathaccent"0\mthex@@81 }%
 \def\hmwwcheck{\mathaccent"0\mthex@@7D }%
 \def\hmwidehat{\mathaccent"0\mthex@@62 }%
 \def\hmwidetilde{\mathaccent"0\mthex@@65 }%
 \def\hmwidecheck{\mathaccent"0\mthex@@7A }%
 \ifx\amstexloaded@\relax
  \Makeacc@\hmWbar{\mthsy@@78}%
  \Makeacc@\hmWwbar{\mthsy@@53}%
  \Makeacc@\hmWhat{\mthsy@@79}%
  \Makeacc@\hmWtilde{\mthsy@@7A}%
  \Makeacc@\hmWcheck{\mthsy@@7B}%
  \Makeacc@\hmWwhat{\mthex@@80}%
  \Makeacc@\hmWwtilde{\mthex@@81}%
  \Makeacc@\hmWwcheck{\mthex@@7D}%
 \fi
 \mathchardef\hmcoprod="1\mthex@@60
 \mathchardef\hmbigvee="1\mthex@@57
 \mathchardef\hmbigwedge="1\mthex@@56
 \mathchardef\hmbiguplus="1\mthex@@55
 \mathchardef\hmbigcap="1\mthex@@54
 \mathchardef\hmbigcup="1\mthex@@53
 \mathchardef\hmintop="1\mthex@@52 \def\hmint{\hmintop\nolimits}%
 \mathchardef\hmprod="1\mthex@@51
 \mathchardef\hmsum="1\mthex@@50
 \mathchardef\hmbigotimes="1\mthex@@4E
 \mathchardef\hmbigoplus="1\mthex@@4C
 \mathchardef\hmbigodot="1\mthex@@4A
 \mathchardef\hmointop="1\mthex@@48 \def\hmoint{\ointop\nolimits}%
 \mathchardef\hmointop="1\mthex@@48 \def\hmoint{\ointop\nolimits}%
 \mathchardef\hmiintop="1\mthex@@92 \def\hmiint{\hmiintop\nolimits}%
 \mathchardef\hmiiintop="1\mthex@@94 \def\hmiiint{\hmiiintop\nolimits}%
 \mathchardef\hmoiintop="1\mthex@@96 \def\hmoiint{\hmoiintop\nolimits}%
 \mathchardef\hmoiiintop="1\mthex@@98 \def\hmoiiint{\hmoiiintop\nolimits}%
 \mathchardef\hmcwointop="1\mthex@@9A \def\hmcwoint{\hmcwointop\nolimits}%
 \mathchardef\hmawointop="1\mthex@@9C \def\hmawoint{\hmawointop\nolimits}%
 \mathchardef\hmcwintop="1\mthex@@9E \def\hmcwint{\hmcwintop\nolimits}%
 \mathchardef\hmbigsqcup="1\mthex@@46
 \def\hmlmoustache{\delimiter"4\mthex@@7A\mthex@@40 }%
 \def\hmrmoustahce{\delimiter"5\mthex@@7B\mthex@@41 }% 
 \def\hmlgroup{\delimiter"4\mthsy@@2E\mthex@@3A }% 
 \def\hmrgroup{\delimiter"5\mthsy@@2F\mthex@@3B }%
 \def\hmbracevert{\delimiter"\mthex@@3E\mthex@@3E }%
 \def\hmchoose{\atopwithdelims\hmlparens\hmrparens}%
 \def\hmbrack{\atopwithdelims\hmlbrack\hmrbrack}%
 \def\hmbrace{\atopwithdelims\hmlbrace\hmrbrace}%
}
\def\heavymath{%
 \textfont\@ne\the\textfont\z@
 \scriptfont\@ne\the\scriptfont\z@
 \scriptscriptfont\@ne\the\scriptscriptfont\z@
 \textfont\tw@\the\textfont\mthsy@
 \scriptfont\tw@\the\scriptfont\mthsy@
 \scriptscriptfont\tw@\the\scriptscriptfont\mthsy@
 \textfont\thr@@\the\textfont\mthex@
 \scriptfont\thr@@\the\scriptfont\mthex@
 \scriptscriptfont\thr@@\the\scriptscriptfont\mthex@
 \ifx\p@renwd\undefined
  \else
  \setbox\z@\hbox{\the\textfont\mthex@ B}\p@renwd\wd\z@
 \fi
 \ifx\amstexloaded@\relax
  \buffer@\fontdimen13\the\textfont\mthex@
  \buffer\buffer@
 \fi
 \let\lmoustache\hmlmoustache
 \let\rmoustache\hmrmoustahce 
 \let\lgroup\hmlgroup
 \let\rgroup\hmrgroup
 \let\bracevert\hmbracevert
 \let\SQRT\sqrt
 \def\ROOT##1\OF##2{\root##1\of{##2}}%
 \def\PARENS##1{\left(##1\right)}%
 \def\LEFTRIGHT##1##2##3{\left##1##2\right##3}%
 \let\widehat\hmwidehat % these 3 wide accents have different definition for mtexa !
 \let\widetilde\hmwidetilde
 \let\widecheck\hmwidecheck
}

% Now come the macros for dealing with the multiple extension fonts.
% First comes a macro to provide proper dimensions and definitions when
% fonts are loaded by hand.

\def\usingMTPextensions#1#2#3#4{\let\MTEXA@#1\let\MTEXE@#2\let\MTEXF@#3\let\MTEXG@##4\relax
 \ifx\p@renwd\undefined
 \else
  \setbox\z@\hbox{#1B}\p@renwd\wd\z@
 \fi
 \ifx\amstexloaded@\relax
  \buffer@\fontdimen13#1%
  \buffer\buffer@
 \fi}

% The following macros presume that \MTEXA@, \MTEXE@, \MTEXF@, and \MTEXG@ can
% be used to refer to the four extension fonts that have been loaded.

\newbox\prePbox@
\newbox\Pbox@
\newif\ifPEX@
\def\PEX@#1{\setbox\Pbox@\vbox{$$\left.\vcenter{\copy\prePbox@}\right)$$}%
 \setbox\Pbox@\vbox{\unvbox\Pbox@\unskip\unpenalty
 \setbox\Pbox@\lastbox
 \setbox\Pbox@\hbox{\unhbox\Pbox@\setbox\Pbox@\lastbox  
 \setbox\Pbox@\hbox{\unhbox\Pbox@\setbox\Pbox@\lastbox  
 \setbox0\hbox{#1}%
 \ifdim\dp\Pbox@>\dp0\global\PEX@true\else
 \global\PEX@false\fi}}}}
\def\EXtest@#1{\setbox\prePbox@\hbox{$\displaystyle{#1}$}%
 \PEX@{\MTEXA@\char32}%
 \ifPEX@ 
  {\textfont3=\MTEXE@\PEX@{\MTEXE@\char12}}%
  \ifPEX@
   {\textfont3=\MTEXF@\PEX@{\MTEXF@\char12}}%
   \ifPEX@
    \def\EXtest@@{\textfont3=\MTEXG@}%
   \else
    \def\EXtest@@{\textfont3=\MTEXF@}%
   \fi
  \else
   \def\EXtest@@{\textfont3=\MTEXE@}%
  \fi
 \else
  \def\EXtest@@{\textfont3=\MTEXA@}%
 \fi}
\def\vc@nt@r#1{\hbox{$\vcenter{\hbox{$\displaystyle{#1}$}}$}}
\newdimen\extcorrect@
\newdimen\vertcorrect@
\extcorrect@\z@
\vertcorrect@\z@
\def\extcorrect#1{\extcorrect@#1\relax}
\def\vertcorrect#1{\vertcorrect@#1\relax\extcorrect@\z@}
\newbox\LRbox@
\def\LEFTRIGHT@#1#2#3{\setbox\LRbox@\vc@nt@r{#3}%
 \EXtest@{\vc@nt@r{#3}}%
 \vcenter{\hbox{\EXtest@@$\displaystyle\left#1\box\LRbox@\right#2$}}}%
\def\PARENS#1{\ifdim\vertcorrect@=\z@\LEFTRIGHT@(){#1}\else\LEFTRIGHT(){#1}\fi}%
\newif\ifspecdelim@
\def\specdelim@#1{\ifx#1(\specdelim@true
 \else\ifx#1)\specdelim@true
 \else\ifx#1<\specdelim@true
 \else\ifx#1\langle\specdelim@true
 \else\ifx#1>\specdelim@true
 \else\ifx#1\rangle\specdelim@true
 \else\ifx#1/\specdelim@true
 \else\ifx#1\backslash\specdelim@true
 \else\specdelim@false\fi\fi\fi\fi\fi\fi\fi\fi}
\def\LEFTRIGHT#1#2#3{%
 \specdelim@#1%
 \ifspecdelim@
  \ifdim\vertcorrect@=\z@
   \LEFTRIGHT@#1.{\vc@nt@r{#3}}%
  \else
   \LEFTRIGHT@#1.{\vc@nt@r{\vrule height\vertcorrect@ depth\z@ width\z@}}%
  \fi
 \else
  \left#1
   \ifdim\extcorrect@=\z@
    \vc@nt@r{#3}%
   \else
    \smash{\vc@nt@r{#3}}\vc@nt@r{\vrule height\extcorrect@ depth\z@ width\z@}%
   \fi
  \right.%
 \fi
 \kern-2\nulldelimiterspace\mskip-\thinmuskip
 \specdelim@#2%
 \ifspecdelim@
  \ifdim\vertcorrect@=\z@
   \LEFTRIGHT@.#2{\vphantom{\vc@nt@r{#3}}}%
  \else
   \LEFTRIGHT@.#2{\vc@nt@r{\vrule height\vertcorrect@ depth\z@ width\z@}}%
  \fi
 \else
  \left.%
  \ifdim\extcorrect@=\z@
   \vphantom{\vc@nt@r{#3}}%
  \else
   \vc@nt@r{\vrule height\extcorrect@ depth\z@ width\z@}%
  \fi
  \right#2%
 \fi
 \extcorrect@\z@\vertcorrect@\z@}%
\newbox\HATbox@
\def\widehat#1{\setbox\HATbox@\hbox{$\displaystyle{#1}$}%
 \setbox0\hbox{\MTEXF@;}%
 \ifdim\wd\HATbox@>\wd0
  \def\HAT@{\textfont3=\MTEXG@}%
 \else
  \setbox0\hbox{\MTEXE@9}%
  \ifdim\wd\HATbox@>\wd0
   \def\HAT@{\textfont3=\MTEXF@}%
  \else
   \setbox0\hbox{\MTEXA@ d}%
   \ifdim\wd\HATbox@>\wd0
    \def\HAT@{\textfont3=\MTEXE@}%
   \else 
    \def\HAT@{\textfont3=\MTEXA@}%
   \fi
  \fi
 \fi
 \hbox{\HAT@$\mathaccent"0362 {#1}$}}%
\newbox\TDbox@
\def\widetilde#1{\setbox\TDbox@\hbox{$\displaystyle{#1}$}%
 \setbox0\hbox{\MTEXF@ K}%
 \ifdim\wd\TDbox@>\wd0
  \def\TD@{\textfont3=\MTEXG@}%
 \else
  \setbox0\hbox{\MTEXE@ I}%
  \ifdim\wd\TDbox@>\wd0
   \def\TD@{\textfont3=\MTEXF@}%
  \else
   \setbox0\hbox{\MTEXA@ d}%
   \ifdim\wd\TDbox@>\wd0
    \def\TD@{\textfont3=\MTEXE@}%
   \else 
    \def\TD@{\textfont3=\MTEXA@}%
   \fi
  \fi
 \fi
 \hbox{\TD@$\mathaccent"0365 {#1}$}}
\newbox\CHbox@
\def\widecheck#1{\setbox\CHbox@\hbox{$\displaystyle{#1}$}%
 \setbox0\hbox{\MTEXF@[}%
 \ifdim\wd\CHbox@>\wd0
  \def\CHECK@{\textfont3=\MTEXG@}%
 \else
  \setbox0\hbox{\MTEXE@ Y}%
  \ifdim\wd\CHbox@>\wd0
   \def\CHECK@{\textfont3=\MTEXF@}%
  \else
   \setbox0\hbox{\MTEXA@ z}%
   \ifdim\wd\CHbox@>\wd0
    \def\CHECK@{\textfont3=\MTEXE@}%
   \else 
    \def\CHECK@{\textfont3=\MTEXA@}%
   \fi
  \fi
 \fi
 \hbox{\CHECK@$\mathaccent"037A {#1}$}}%
\newbox\preSbox@
\newbox\Sbox@
\newif\ifSQEX@
\def\SQEX@#1{\setbox\Sbox@\vbox{$$\radical"270370{\copy\preSbox@}$$}%
 \setbox\Sbox@\vbox{\unvbox\Sbox@\unskip\unpenalty
 \setbox\Sbox@\lastbox\setbox\Sbox@\hbox{\unhbox\Sbox@\setbox\Sbox@\lastbox
 \setbox\Sbox@\hbox{\unhbox\Sbox@\setbox\Sbox@\lastbox\setbox\Sbox@\lastbox
 \setbox0\hbox{#1}%
 \ifdim\dp\Sbox@>\dp0\global\SQEX@true\else
 \global\SQEX@false\fi}}}}
\newcount\SQcount@
\def\SQtest@#1{\setbox\preSbox@\hbox{$\displaystyle{#1}$}%
 \SQEX@{\MTEXA@ s}%
 \ifSQEX@
  {\textfont3=\MTEXE@\SQEX@{\MTEXE@ u}}%
  \ifSQEX@
   {\textfont3=\MTEXF@\SQEX@{\MTEXF@ u}}%
    \ifSQEX@
     \def\SQtest@@{\textfont3=\MTEXG@}\global\SQcount@3
    \else
     \def\SQtest@@{\textfont3=\MTEXF@}\global\SQcount@2
    \fi
  \else
   \def\SQtest@@{\textfont3=\MTEXE@}\global\SQcount@1
  \fi
 \else
  \def\SQtest@@{\textfont3=\MTEXA@}\global\SQcount@0
 \fi}
\newbox\SQRTbox@
\def\SQRT#1{\setbox\SQRTbox@\hbox{$\displaystyle{#1}$}%
 \SQtest@{#1}%
 \hbox{\SQtest@@$\displaystyle\radical"270370{\box\SQRTbox@}$}}
\def\ROOT#1\OF#2{\setbox\rootbox\hbox{$\m@th\scriptscriptstyle{#1}$}%
 \mathpalette\R@@T{#2}}
\def\R@@T#1#2{\setbox\z@\hbox{$\uproot@\z@\leftroot\z@\m@th#1\SQRT{#2}$}%
 \dimen@\ht\z@\advance\dimen@-\dp\z@
 \dimen@ii\dimen@
 \setbox\tw@\hbox{$\m@th#1\mskip\uproot@ mu$}\advance\dimen@ii by1.667\wd\tw@
 \setbox\tw@\hbox{$\m@th#1\mskip10mu$}%\advance\dimen@ by1.667\wd\tw@
 \ifcase\SQcount@\advance\dimen@3\wd\tw@\or\advance\dimen@1.5\wd\tw@\or
 \advance\dimen@\wd\tw@\fi
 \mkern1mu\kern.13\dimen@\mkern-\leftroot@ mu
 \raise.44\dimen@ii\copy\rootbox
 \mkern-1mu\kern-.13\dimen@\mkern\leftroot@ mu\box\z@\kern-\wd\rootbox
 \leftroot\z@\uproot\z@}

% Now come the macros for the extra large symbols.  They assume that
% \MTXL@ and \MTXXXL@ refer to the fonts mtxl and mtxxxl that have been loaded.

\def\eat@#1{}

% \FNSS@ is \futurelet\next skipping spaces before the next token

\def\FNSS@#1{\let\FNSS@@#1\FN@\FNSS@@@}
\def\FNSS@@@{\ifx\next\space@\def\FNSS@@@@. {\FN@\FNSS@@@}\else
 \def\FNSS@@@@.{\FNSS@@}\fi\FNSS@@@@.}

% we define struts for subscripts and superscripts to give the extra space that would
% normally be provided for limits of large operators

\def\fdxiii@sub{\vrule height\fontdimen13 \the\textfont3 width\z@ depth\z@}
\def\fdxiii@sup{\vrule depth\fontdimen13 \the\textfont3 width\z@ height\z@}

% \XL#1#2 etc., will define \next@, depending on the next symbol
% \largeopx@ and \largeopxNL@ will be used for \XL and \XXL, with all operators other 
% than \int, etc., and \largeopxxx@ and \largeopxxxNL@ will be used for \XXXL. 
% We make the definitions in terms of a common one that uses the font as the argument.

\def\LARGEOPx@#1#2#3{% #1=0 or 1 for 'limits' or no 'limits'; #2 is font, #3 is char position
 \ifnum#1=\@ne
  \def\next@{\mathop{\hbox{$\vcenter{\hbox{#2\char#3}}$}}\nolimits
    _{\SUB@}^{\SUP@}\eat@}%
 \else
  \def\next@{\mathop{\hbox{$\vcenter{\hbox{#2\char#3}}$}}%
    _{\SUB@}^{\SUP@}\eat@}%
 \fi}


\def\largeopx@#1#2{\LARGEOPx@#1\MTXL@{#2}}
\def\largeopxxx@#1#2{\LARGEOPx@#1\MTXXXL@{#2}}

% Similarly, for \LARGEINTx@.  When there are 'limits', the construction is especially 
% complicated. \maxXLscripts@ will store the maximum of the widths of the sub 
% and superscripts. There is the additional complication that the amount to adjust the
% superscript differs for \XL and \XXL, and the adjustment is made in terms of an extra
% \fontdimen in the mtxxl font, which measures the horizontal distance between the
% lowest and highest points of the integral sign (for the \XXL versions these are exactly
% twice the \XL versions).


\newdimen\maxXLscripts@
\newcount\X@count % 0 for \XL, 1 for \XXL, 2 for \XXXL, 3 for \xl

\def\LARGEINTx@#1#2#3{% 
 \ifnum#1=\@ne
  \def\next@{\setbox\z@\hbox{#2\char#3\/}\dimen@\wd\z@
  \setbox\z@\hbox{#2\char#3}\advance\dimen@-\wd\z@
  \mathop{\hbox{$\vcenter{\hbox{#2\char#3}}$}}\nolimits
   _{\SUB@}^{\kern\dimen@\SUP@}\eat@}%
 \else                             
  \def\next@{\setbox\z@\hbox{\ifcase\X@count\kern\tw@\fontdimen8\MTXL@\or
   \kern4\fontdimen8\MTXL@\or\kern\tw@\fontdimen8\MTXXXL@\or\kern1.7\fontdimen8\MTXL@\fi}%
  \setbox\@ne\hbox{#2\char#3}%
  \setbox\tw@\hbox{$\scriptstyle{\SUB@}$}%
  \setbox\thr@@\hbox{$\kern\wd\z@\scriptstyle{\SUP@}$}%
  % let \maxXLscripts@ be max of subscript and superscript boxes
  \maxXLscripts@\wd\thr@@\ifdim\maxXLscripts@<\wd\tw@\maxXLscripts@\wd\tw@\fi
 % let \dimen@ii be amount of subscript to left of int 
  \dimen@ii.5\wd\tw@ \advance\dimen@ii-.5\wd\@ne
 % let \dimen@ be amount of visible superscript to left of int, namely
    % [visible length] - [amount to right of left boundary of \int sign], i.e.,
    %% [wd3 - \wd0] - 1/2(wd3 + wd1) 
  \dimen@.5\wd\thr@@ \advance\dimen@-\wd\z@ \advance\dimen@-.5\wd\@ne
  \ifdim\dimen@>\z@ % if visible part of superscript extends to left of \int
    \ifdim\dimen@>\dimen@ii  % if visible part of superscript to left of subscript
                             % kern by - [1/2(\maxXLscripts@ - wd1) - \dimen@]
     \kern\dimen@\kern.5\wd\@ne\kern-.5\maxXLscripts@
    \else                    % only trim to subscript,
                             % kern - [1/2(\maxXLscripts@ - wd1) - \dimen@ii]
     \kern\dimen@ii\kern.5\wd\@ne\kern-.5\maxXLscripts@
    \fi
  \else % visible part of superscript entirely to right of \int, so trim to subscript
    \ifdim\dimen@ii > \z@
     \kern\dimen@ii\kern.5\wd\@ne\kern-.5\maxXLscripts@
    \else
     \kern.5\wd\@ne\kern-.5\maxXLscripts@
    \fi 
  \fi
  \setbox\@ne\hbox{#2\char#3\/}\dimen@ii\wd\@ne
  \setbox\@ne\hbox{#2\char#3}\advance\dimen@ii-\wd\@ne
  \mathop{\hbox{$\vcenter{\hbox{#2\char#3}}$}}_{\SUB@}^{\kern\wd\z@\SUP@}\kern\dimen@ii\eat@}%
 \fi}


\def\largeintx@#1#2{\LARGEINTx@#1\MTXL@{#2}}
\def\largeintxxx@#1#2{\LARGEINTx@#1\MTXXXL@{#2}}

\newcount\XLtype@

\def\xl{\XLtype@\z@\x@l}
\def\xlnl{\XLtype@\@ne\x@l}
\def\x@l#1#2{\def\SUB@{#1}\def\SUP@{#2}\futurelet\next\xl@}
\def\xl@{\X@count\thr@@
 \ifx\next\bigodot\largeopx@\XLtype@{96}\else
 \ifx\next\bigoplus\largeopx@\XLtype@{97}\else
 \ifx\next\bigotimes\largeopx@\XLtype@{98}\else
 \ifx\next\bigsqcup\largeopx@\XLtype@{99}\else
 \ifx\next\bigcup\largeopx@\XLtype@{100}\else
 \ifx\next\bigcap\largeopx@\XLtype@{101}\else
 \ifx\next\biguplus\largeopx@\XLtype@{102}\else
 \ifx\next\bigwedge\largeopx@\XLtype@{103}\else
 \ifx\next\bigvee\largeopx@\XLtype@{104}\else
 \ifx\next\sum\largeopx@\XLtype@{105}\else
 \ifx\next\prod\largeopx@\XLtype@{106}\else
 \ifx\next\coprod\largeopx@\XLtype@{107}\else
 \ifx\next\int\largeintx@\XLtype@{108}\else
 \ifx\next\oint\largeintx@\XLtype@{109}\else
 \ifx\next\bigcupprod\largeopx@\XLtype@{110}\else
 \ifx\next\bigcapprod\largeopx@\XLtype@{111}\else
 \ifx\next\cwoint\largeintx@\XLtype@{112}\else
 \ifx\next\awoint\largeintx@\XLtype@{113}\else
 \ifx\next\cwint\largeintx@\XLtype@{114}\else
 \ifx\next\iint\largeintx@\XLtype@{115}\else
 \ifx\next\iiint\largeintx@\XLtype@{116}\else
 \ifx\next\oiint\largeintx@\XLtype@{117}\else
 \ifx\next\oiiint\largeintx@\XLtype@{118}\else
 \errmessage{Invalid use of \noexpand\xl}%
 \fi\fi\fi\fi\fi\fi\fi\fi\fi\fi\fi\fi\fi\fi\fi\fi\fi\fi\fi\fi\fi\fi\fi\next@}




\def\XL{\XLtype@\z@\X@L}
\def\XLNL{\XLtype@\@ne\X@L}
\def\X@L#1#2{\def\SUB@{#1}\def\SUP@{#2}\futurelet\next\XL@}
\def\XL@{\X@count\z@
 \ifx\next\bigodot\largeopx@\XLtype@0\else
 \ifx\next\bigoplus\largeopx@\XLtype@1\else
 \ifx\next\bigotimes\largeopx@\XLtype@2\else
 \ifx\next\bigsqcup\largeopx@\XLtype@3\else
 \ifx\next\bigcup\largeopx@\XLtype@4\else
 \ifx\next\bigcap\largeopx@\XLtype@5\else
 \ifx\next\biguplus\largeopx@\XLtype@6\else
 \ifx\next\bigwedge\largeopx@\XLtype@7\else
 \ifx\next\bigvee\largeopx@\XLtype@8\else
 \ifx\next\sum\largeopx@\XLtype@9\else
 \ifx\next\prod\largeopx@\XLtype@{10}\else
 \ifx\next\coprod\largeopx@\XLtype@{11}\else
 \ifx\next\int\largeintx@\XLtype@{12}\else
 \ifx\next\oint\largeintx@\XLtype@{13}\else
 \ifx\next\bigcupprod\largeopx@\XLtype@{14}\else
 \ifx\next\bigcapprod\largeopx@\XLtype@{15}\else
 \ifx\next\cwoint\largeintx@\XLtype@{16}\else
 \ifx\next\awoint\largeintx@\XLtype@{17}\else
 \ifx\next\cwint\largeintx@\XLtype@{18}\else
 \ifx\next\iint\largeintx@\XLtype@{19}\else
 \ifx\next\iiint\largeintx@\XLtype@{20}\else
 \ifx\next\oiint\largeintx@\XLtype@{21}\else
 \ifx\next\oiiint\largeintx@\XLtype@{22}\else
 \errmessage{Invalid use of \noexpand\XL}%
 \fi\fi\fi\fi\fi\fi\fi\fi\fi\fi\fi\fi\fi\fi\fi\fi\fi\fi\fi\fi\fi\fi\fi\next@}

\def\XXL{\XLtype@\z@\XX@L}
\def\XXLNL{\XLtype@\@ne\XX@L}
\def\XX@L#1#2{\def\SUB@{#1}\def\SUP@{#2}\futurelet\next\XXL@}
\def\XXL@{\X@count\@ne
 \ifx\next\bigodot\largeopx@\XLtype@{48}\else
 \ifx\next\bigoplus\largeopx@\XLtype@{49}\else
 \ifx\next\bigotimes\largeopx@\XLtype@{50}\else
 \ifx\next\bigsqcup\largeopx@\XLtype@{51}\else
 \ifx\next\bigcup\largeopx@\XLtype@{52}\else
 \ifx\next\bigcap\largeopx@\XLtype@{53}\else
 \ifx\next\biguplus\largeopx@\XLtype@{54}\else
 \ifx\next\bigwedge\largeopx@\XLtype@{55}\else
 \ifx\next\bigvee\largeopx@\XLtype@{56}\else
 \ifx\next\sum\largeopx@\XLtype@{57}\else
 \ifx\next\prod\largeopx@\XLtype@{58}\else
 \ifx\next\coprod\largeopx@\XLtype@{59}\else
 \ifx\next\int\largeintx@\XLtype@{60}\else
 \ifx\next\oint\largeintx@\XLtype@{61}\else
 \ifx\next\bigcupprod\largeopx@\XLtype@{62 \char64}\else
 \ifx\next\bigcapprod\largeopx@\XLtype@{63 \char65}\else
 \ifx\next\cwoint\largeintx@\XLtype@{66}\else
 \ifx\next\awoint\largeintx@\XLtype@{67}\else
 \ifx\next\cwint\largeintx@\XLtype@{68}\else
 \ifx\next\iint\largeintx@\XLtype@{69}\else
 \ifx\next\iiint\largeintx@\XLtype@{70}\else
 \ifx\next\oiint\largeintx@\XLtype@{71}\else
 \ifx\next\oiiint\largeintx@\XLtype@{72}\else
 \errmessage{Invalid use of \noexpand\XXL}%
 \fi\fi\fi\fi\fi\fi\fi\fi\fi\fi\fi\fi\fi\fi\fi\fi\fi\fi\fi\fi\fi\fi\fi\next@}

\def\XXXL{\XLtype@\z@\XXX@L}
\def\XXXLNL{\XLtype@\@ne\XXX@L}
\def\XXX@L#1#2{\def\SUB@{#1}\def\SUP@{#2}\futurelet\next\XXXL@}
\def\XXXL@{\X@count\tw@
 \ifx\next\bigodot\largeopxxx@\XLtype@0\else
 \ifx\next\bigoplus\largeopxxx@\XLtype@1\else
 \ifx\next\bigotimes\largeopxxx@\XLtype@2\else
 \ifx\next\bigsqcup\largeopxxx@\XLtype@3\else
 \ifx\next\bigcup\largeopxxx@\XLtype@4\else
 \ifx\next\bigcap\largeopxxx@\XLtype@5\else
 \ifx\next\biguplus\largeopxxx@\XLtype@6\else
 \ifx\next\bigwedge\largeopxxx@\XLtype@7\else
 \ifx\next\bigvee\largeopxxx@\XLtype@8\else
 \ifx\next\sum\largeopxxx@\XLtype@9\else
 \ifx\next\prod\largeopxxx@\XLtype@{10}\else
 \ifx\next\coprod\largeopxxx@\XLtype@{11}\else
 \ifx\next\int\largeintxxx@\XLtype@{12}\else
 \ifx\next\oint\largeintxxx@\XLtype@{13}\else
  % xxxl \bigcupprod is too wide for PostScript, so it is split into parts
  % in positions 14 and 16; similarly \bigcapprod is in positions 15 and 17
 \ifx\next\bigcupprod\largeopxxx@\XLtype@{14 \char16}\else
 \ifx\next\bigcapprod\largeopxxx@\XLtype@{15 \char17}\else
 \ifx\next\cwoint\largeintxxx@\XLtype@{18}\else
 \ifx\next\awoint\largeintxxx@\XLtype@{19}\else
 \ifx\next\cwint\largeintxxx@\XLtype@{20}\else
 \ifx\next\iint\largeintxxx@\XLtype@{21}\else
 \ifx\next\iiint\largeintxxx@\XLtype@{22}\else
 \ifx\next\oiint\largeintxxx@\XLtype@{23}\else
 \ifx\next\oiiint\largeintxxx@\XLtype@{24}\else
 \def\next@{\errmessage{Invalid use of \noexpand\XXXL}}%
 \fi\fi\fi\fi\fi\fi\fi\fi\fi\fi\fi\fi\fi\fi\fi\fi\fi\fi\fi\fi\fi\fi\fi\next@}

%%%%%%%%%%%%%%%%%%%%%%%%%%%%%%%%%%%%%%%%%%%%%%%%%%%%%%%%%%%%%%%%%%%%%%
%% If designing your own style file, you will probably want to omit %%
%% everything from here to the end. Remember to  return @ and " to  %%
%% their proper category codes, if necessary.                       %%
%%%%%%%%%%%%%%%%%%%%%%%%%%%%%%%%%%%%%%%%%%%%%%%%%%%%%%%%%%%%%%%%%%%%%%

\def\FONT@#1#2{\expandafter\ifx\csname#1#2\endcsname\relax
 \expandafter\expandafter\expandafter\global
 \expandafter\font\csname#1#2\endcsname=#1#2\fi
 \def\next@{\let\next@}%
 \expandafter\next@\csname#1#2\endcsname}
\def\PSZ@{\edef\nextiii@{ at \the\dimen@}}
\def\MTPMI@#1#2#3{%
 \dimen@#1\relax\PSZ@
 \FONT@{mtmit}\nextiii@\textfont\@ne\next@\skewchar\next@45
 \dimen@#2\relax\PSZ@
 \FONT@{mtmis}\nextiii@\scriptfont\@ne\next@\skewchar\next@45
 \dimen@#3\relax\PSZ@
 \FONT@{mtmif}\nextiii@\scriptscriptfont\@ne\next@\skewchar\next@45\relax
}
\def\MTPSY@#1#2#3{%
 \dimen@#1\relax\PSZ@
 \FONT@{mtsyt}\nextiii@\textfont\tw@\next@\skewchar\next@48
 \dimen@#2\relax\PSZ@
 \FONT@{mtsys}\nextiii@\scriptfont\tw@\next@\skewchar\next@48
 \dimen@#3\relax\PSZ@
 \FONT@{mtsyf}\nextiii@\scriptscriptfont\tw@\next@\skewchar\next@48\relax
}
\def\MTPEX@#1{%
 \dimen@#1\relax\PSZ@
 \FONT@{mtexa}\nextiii@
 \let\MTEXA@\next@
 \setbox0\hbox{\MTEXA@\char'165}%
 \textfont\thr@@\next@
 \scriptfont\thr@@\next@
 \scriptscriptfont\thr@@\next@
 \ifx\p@renwd\undefined
 \else
  \setbox\z@\hbox{\next@ B}\p@renwd\wd\z@
 \fi
 \ifx\amstexloaded@\relax
  \buffer@\fontdimen13 \next@
  \buffer\buffer@
 \fi
 \FONT@{mtxl}\nextiii@
 \let\MTXL@\next@
 \multiply\dimen@\tw@\PSZ@\FONT@{mtexe}\nextiii@\let\MTEXE@\next@
  \FONT@{mtxxxl}\nextiii@\let\MTXXXL@\next@
 \multiply\dimen@\tw@\PSZ@\FONT@{mtexf}\nextiii@\let\MTEXF@\next@
 \multiply\dimen@\tw@\PSZ@\FONT@{mtexg}\nextiii@\let\MTEXG@\next@
}
\def\MTPMB@#1#2#3{%
 \dimen@#1\relax\PSZ@
 \FONT@{mtmbt}\nextiii@\textfont\mbffam\next@\skewchar\next@32
 \dimen@#2\relax\PSZ@
 \FONT@{mtmbs}\nextiii@\scriptfont\mbffam\next@\skewchar\next@32
 \dimen@#3\relax\PSZ@
 \FONT@{mtmbf}\nextiii@\scriptscriptfont\mbffam\next@\skewchar\next@32
}
\def\mbf#1{{\fam\mbffam#1}}
\def\MTPBMI@#1#2#3{%
 \alloc@@8\fam\chardef\sixt@@n\mtbmi@
 \edef\mtbmi@@{\hexnumber@\mtbmi@}%
 \dimen@#1\relax\PSZ@
 \FONT@{mtbmit}\nextiii@\textfont\mtbmi@\next@\skewchar\next@45
 \dimen@#2\relax\PSZ@
 \FONT@{mtbmis}\nextiii@\scriptfont\mtbmi@\next@\skewchar\next@45
 \dimen@#3\relax\PSZ@
 \FONT@{mtbmif}\nextiii@\scriptscriptfont\mtbmi@\next@\skewchar\next@45\relax
}
\def\MTPBSY@#1#2#3{%
 \alloc@@8\fam\chardef\sixt@@n\mtbsy@
 \edef\mtbsy@@{\hexnumber@\mtbsy@}%
 \dimen@#1\relax\PSZ@
 \FONT@{mtbsyt}\nextiii@\textfont\mtbsy@\next@\skewchar\next@48
 \dimen@#2\relax\PSZ@
 \FONT@{mtbsys}\nextiii@\scriptfont\mtbsy@\next@\skewchar\next@48
 \dimen@#3\relax\PSZ@
 \FONT@{mtbsyf}\nextiii@\scriptscriptfont\mtbsy@\next@\skewchar\next@48\relax
}
\def\MTPBEX@#1{%
 \alloc@@8\fam\chardef\sixt@@n\mtbex@
 \edef\mtbex@@{\hexnumber@\mtbex@}%
 \dimen@#1\relax\PSZ@
 \FONT@{mtbexa}\nextiii@
 \textfont\mtbex@\next@
 \scriptfont\mtbex@\next@
 \scriptscriptfont\mtbex@\next@
}
\def\MTPHSY@#1#2#3{%
 \alloc@@8\fam\chardef\sixt@@n\mthsy@
 \edef\mthsy@@{\hexnumber@\mthsy@}%
 \dimen@#1\relax\PSZ@
 \FONT@{mthsyt}\nextiii@\textfont\mthsy@\next@\skewchar\next@48
 \dimen@#2\relax\PSZ@
 \FONT@{mthsys}\nextiii@\scriptfont\mthsy@\next@\skewchar\next@48
 \dimen@#3\relax\PSZ@
 \FONT@{mthsyf}\nextiii@\scriptscriptfont\mthsy@\next@\skewchar\next@48\relax
}
\def\MTPHEX@#1{%
 \alloc@@8\fam\chardef\sixt@@n\mthex@
 \edef\mthex@@{\hexnumber@\mthex@}%
 \dimen@#1\relax\PSZ@
 \FONT@{mthexa}\nextiii@
 \textfont\mthex@\next@
 \scriptfont\mthex@\next@
 \scriptscriptfont\mthex@\next@
}
\def\CAL@#1#2#3{%
 \usecal
 \FONT@{#1}\empty\textfont\Calfam\next@\skewchar\next@48
 \FONT@{#2}\empty\scriptfont\Calfam\next@\skewchar\next@48
 \FONT@{#3}\empty\scriptscriptfont\Calfam\next@\skewchar\next@48\relax
}
\def\MTPsizes#1#2#3{%
 \def\tMTPsize{#1}\def\sMTPsize{#2}\def\fMTPsize{#3}%  % info for mtpb.tex and mtph.tex
 \ifx\timesmt@loaded\relax 
  \TimesMTsizes{#1}{#2}{#3}%
 \fi
 \MTPMI@{#1}{#2}{#3}%
 \MTPSY@{#1}{#2}{#3}%
 \MTPEX@{#1}%
 \MTPMB@{#1}{#2}{#3}%
 \CAL@{cmsy10 at#1}{cmsy7 at#2}{cmsy5 at#3}%
 \usingMTPsizes{#1}{#2}{#3}%
}
\def\loadbm{\MTPBMI@\tMTPsize\sMTPsize\fMTPsize
 \MTPBSY@\tMTPsize\sMTPsize\fMTPsize
 \MTPBEX@\tMTPsize
 \bmdefs@}
\def\loadhm{\MTPHSY@\tMTPsize\sMTPsize\fMTPsize
 \MTPHEX@\tMTPsize
 \hmdefs@}

\catcode`\"=\qqcode@
\catcode`\@=\atcode@
\MTPsizes{10pt}{7pt}{5.5pt}



 